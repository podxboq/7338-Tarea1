Supongamos que tenemos un Grupo de Lie, que está formado por las distintas transformaciones que se le pueden hacer a un sistema y le llamamos "Grupo de Transformación".
Si tenemos unas coordenadas originales:
\begin{equation}
    x = (x_1, x_2, ..., x_n)
    u = u(x) = (u_1(x), u_2(x), ..., u_n(x))
\end{equation}

Y las coordenadas correspondientes de aplicar una transformación a las originales:
\begin{equation}
    y = (y_1, y_2, ..., y_n)
    v = v(x) = (v_1(x), v_2(x), ..., v_n(x))
\end{equation}

Llamaremos "Invariante" a cualquier función que cumple que, aplicada a los parámetros y las funciones (y sus derivadas) originales sea igual a la misma aplicación a las transformadas:
\begin{equation}
    P(x,u,\frac{\partial u}{\partial x}, \frac{\partial^2 u}{\partial^2 x}, ...) = P(y, v, \frac{\partial v}{\partial y}, \frac{\partial^2 v}{\partial y}, ...)
\end{equation}

En concreto, una integral $I$, sobre un intervalo arbitrario de $x$ y su correspondiente intervalo de $y$, será invariante, si existe una relación:
\begin{equation}
    I = \int_I ...\int_I f(x,u,\frac{\partial u}{\partial x}, \frac{\partial^2 u}{\partial^2 x}, ...) dx = \int_I ...\int_I f(y, v, \frac{\partial v}{\partial y}, \frac{\partial^2 v}{\partial y}, ...)
\end{equation}

Queremos confirmar que la integral I es invariante, así que vamos a calcular $\delta I$ en función de las coordenadas originales:
\begin{equation}
    \delta I = \int_I ... \int_I f dx = \int_I ... \int_I \delta f(x,u,\frac{\partial u}{\partial x}, \frac{\partial^2 u}{\partial^2 x}, ...) dx 
\end{equation}

Para quitarnos los límites de la integral, introducimos la variación $\delta u$ dentro de la misma:
\begin{equation}
    \delta I = \int ... \int f dx = \int ... \int \delta \sum_i^n \psi_i(x,u,\frac{\partial u}{\partial x}, \frac{\partial^2 u}{\partial^2 x}, ...) \delta u_i dx 
\end{equation}

Donde \psi indica cada una de las expresiones del lagrangiano para cada conjunto de parámetros funcionales.