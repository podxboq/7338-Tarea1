\begin{abstract}
	En este trabajo vamos a exponer el concepto de simetría en física y de como éstas son aplicadas a las ecuaciones del movimiento, para obtener unos resultados sorprendentes sobre conceptos fundamentales en la física, como al conservación de la energía.

	Aunque tras explicar la relación entre simetría y conservación, gracias al teorema de Noether, vamos a ser capaces de encontrar propiedades aún más fundamentales de la física, en concreto, de la física cuántica.

	Veremos como la existencia de las simetrías puede ser codificado matemáticamente en el concepto de grupo de simetría y de como los modelos en física de partículas vienen al final caracterizados por dichos grupos.

	Finalmente nos adentraremos brevemente por el modelo estándar para entender como los grupos de simetrías configuran las partículas y sus interacciones.
\end{abstract}