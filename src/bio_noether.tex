\chapter{Emmy Noether}\label{ch:emmy-noether}

\section{Biografía}\label{sec:biografia}

No se puede hablar del estudio de las simetrías sin hacerlo de \define{Amalie Emmy Noether}{Emmy Noether}, nacida el 23 de Marzo de 1882 en Erlangen, Alemania.

Procedente de una familia de confesión Judía, el padre de Emmy, \define{Max Noether}{Max Noether}, fue un reconocido matemático autodidacta que consiguió un puesto en la Universidad de Erlangen.

A pesar de ello, Emmy no mostró especial interes en las matemáticas de joven.
Recibió, de su madre, una educación acorde a lo que le esperaba a una mujer en su tiempo (cocina y demás tareas del hogar) y también clases de clavecín y danza (sin mucho entusiasmo por su parte).
Le gustaba frecuentar los bailes que organizaban los amigos de su padre y tenía más interes en aprender lenguas.
Tanto es así, que se preparó y superó los exámenes para ejercer de profesora de Inglés y Francés para los colegios de niñas en el estado de Baviera (\cite{Carrasco}).

Tras eso, quiso proseguir con sus estudios de lenguas en la universidad, sin embargo, a las mujeres no les estaba permitido matricularse de forma ordinaria en la universidad de Erlangen.
Para ir a las clases tendría que hacerlo de forma extraoficial y con el consentimiento de cada profesor, aunque sí que podría presentarse a los exámenes que certificaban su paso por la universidad (\cite{Kimberling}).
Durante sus estudios universitarios, decidió pasarse de la lengua a las matemáticas, y en 1903, se examina en Nuremberg bajo la dirección del matemático Aurel Vo\beta (\cite{Kimberling}).

En 1903, Emmy empieza sus clases en la Universidad de Gotinga, impartidas por eminentes matemáticos, entre otros, \define{David Hilbert}{David Hilbert} que se convertiría en su principal apoyo contra las trabas que, como mujer, tendría en el entorno académico.

Pero seis meses después, volvería a Erlangen, pues ya permitían matricularse a las mujeres (\cite{Kimberling}).

En 1907, presenta su tesis doctoral \textquote{Über die Bildung des Formensystems der ternären biquadratischen Form} (Sobre los sistemas completos de invariantes para formas ternarias bicuadráticas), bajo la supervisión de \define{Paul Gordan}{Paul Gordan}, matemático que se había especializado en invariantes.

En esa época, las mujeres tampoco tenían acceso a la posición de profesoras remuneradas en la universidad, por lo que Emmy tuvo que contentarse con dar clases sustituyendo a su padre cuando estaba enfermo.
Aún así, continuó sus investigaciones acerca de los invariantes por su cuenta.

Durante este periodo fue muy relevante el cambio de la dirección de Paul Gordan por la de \define{Ernst Fischer}{Ernst Fischer}, dada la reticencia de Gordan a los métodos de generalización de \define{Hilbert}{David Hilbert}, a los que denominó \textquotedblleft Teología\textquotedblright\ con cierta sorna, lo que permitió a Noether disponer de más herramientas para sus investigaciones (\cite{Weyl}).

En este periodo es cuando empieza a tener renombre gracias a sus publicaciones.

Hilbert estaba interesado en los conocimientos y habilidades de Noether en el cálculo de invariantes para aplicarlo a sus trabajos sobre la \define{Teoría de la Relatividad General}{Teoría de la Relatividad General}\index{Teoría de la Relatividad General},
por lo que él y Felix Klein, invitaron a Noether en 1915 a trabajar con ellos en la Universidad de Gotinga.
Aquí, una vez más, Noether sufrió la discriminación al ser privada de un puesto remunerado como profesora (conocido como \textit{privatdozent}) por ser mujer.
Aunque Hilbert intentó en varias ocasiones revertir esta situación ante el claustro de la Facultad, no tuvo éxito y en una de ellas espetó: \textquote{No veo que el sexo de la candidata sea un argumento contra su admisión como privatdozent. Después de todo, somos una universidad y no una casa de baños} (\cite[332]{Carrasco}).

A pesar de la importancia de sus trabajos, Emmy Noether no tendría el reconocimiento academico que se merecía.
Como comenta Einstein en una carta a \define{Hilbert}{David Hilbert}: \textquote{Ayer recibí un artículo muy interesante de la señorita Noether sobre la formación de invariantes. Me impresiona que este tipo de cosas puedan ser comprendidas de un modo tan general. La vieja guardia de Gotingen debería de tomar algunas lecciones de la señorita Noether. ¡Parece que sabe lo que hace!} (\cite{Kimberling}).

Tras el final de la \define{Primera Guerra Mundial}{Primera Guerra Mundial}, estalló en Alemánia la \define{República de Weymar}{República de Weymar}, lo que supuso una paulatina mejora en las condiciones de acceso de las mujeres a las universidades (además del sufragio femenino).
Así, Noether consigió ser reconocida oficialmente como profesora, aunque le costaría aun tres años que este reconocimiento conllevase una retribución económica (\cite[333]{Carrasco}).

Durante este periodo de tiempo, el trabajo de Noether fue clave para la formulación de la \define{Teoría de la Relatividad General}{Teoría de la Relatividad General}\index{Teoría de la Relatividad General} (\cite{Weyl}), transcendería la fama de Noether en el mundo académico y obtuvo el reconocimiento que se merecía.
Tuvo alumnos que vinieron desde diversas partes del mundo para estudiar con ella, a los que se concía como los \textquotedblleft los chicos de Noether\textquotedblright\ (\cite{Carrasco}).

Sin embargo, años después, la cosas se complicarían.
La llegada de \define{Hitler}{Hitler}\index{Hitler} al poder, supuso la expulsión de las universidad de las personas judías y Noether tuvo que unirse a la larga lista de brillantes cientificos que abandonaron Alemania por la persecución Nazi (junto con Einstein, Von Neumann, etc), y que tomaron refugio en los \define{Estados Unidos de América}{Estados Unidos de América}\index{Estados Unidos de América} (\cite{Kimberling}).
Como diría Einstein: \textquote{Su significante y altruista trabajo, realizado en un periodo de varios años, fue recompensado por los nuevos governantes de Alemania con un despido, que le costaría a ella la pérdidA de los medios para mantener su vida sencilla y la oportunidad de continuar con sus estudios matemáticos} (\cite{Einstein}).

Una vez en América, trabajó impartiendo clases de matemáticas en el \define{Bryn Mawr College}{Bryn Mawr College}, una universisdad femenina y cooperó con el \define{Instituto para Estudios Avanzados de Princeton}{Princeton}, donde trabajaba en ese momento Einstein (\cite{Carrasco}).

Dos años después, el 14 de Abril de 1935, con 54 años, Emmy fallecería a causa de complicaciones cardiacas en una operación.
