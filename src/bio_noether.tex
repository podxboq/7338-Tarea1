\section{Emmy Noether}
No se puede hablar del estudio de las simetrías sin hacerlo de \define{Amalie Emmy Noether}{Emmy Noether}\index{Emmy Noether}, nacida el 23 de Marzo de 1882 en Erlangen, Alemania.

Procedente de una familia de confesión Judía, el padre de Emmy, \define{Max Noether}{Max Noether}\index{Max Noether}, fue un reconocido matemático autodidacta que consiguió un puesto en la Universidad de Erlangen.

A pesar de ello, Emmy no mostró especial interes en las matemáticas de joven. Recibió, de su madre, una educación acorde a lo que le esperaba a una mujer en su tiempo (cocina y demás tareas del hogar) y también clases de clavecín y danza (sin mucho entusiasmo por su parte). Le gustaba frecuentar los bailes que organizaban los amigos de su padre y tenía más interes en aprender lenguas. Tanto es así, que se preparó y superó los exámente para ejercer de profesora de Inglés y Francés para los colegios de niñas en el estado de Baviera \cite{Carrasco}.

Tras eso, quiso proseguir con sus estudios de lenguas en la universidad, sin embargo, a las mujeres no les estaba permitido matricularse de forma ordinaria en la universidad de Erlangen. Para ir a las clases tendría que hacerlo de forma extraoficial y con el consentimiento de cada profesor, aunque sí que podría presentarse a los exámenes que certificaban su paso por la universidad \cite{Kimberling}. 
Durante sus estudios universitarios, decidió pasarse de la lengua a las matemáticas, y en 1903, se examina en Nuremberg bajo la dirección del matemático Aurel Vo\beta \cite{Kimberling}.

En 1903, Emmy empieza sus clases en la Universidad de Gotinga, impartidas por eminentes matemáticos, entre otros, \define{David Hilbert}{David Hilbert}\index{David Hilbert} que se convertiría en su principal apoyo contra las trabas que, como mujer, tendría en el entorno académico.

Pero seis meses después, volvería a Erlangen, pues ya permitían matricularse a las mujeres (Kimberling 1972).

En 1907, presenta su tesis doctoral "Über die Bildung des Formensystems der ternären biquadratischen Form" (Sobre los sistemas completos de invariantes para formas ternarias bicuadráticas), bajo la supervisiónd de \define{Paul Gordan}{Paul Gordan}\index{Paul Gordan}, matemático que se había especializado en invariantes.

En esa época, las mujeres tampoco tenían acceso a la posición de profesoras remuneradas en la universidad, por lo que Emmy tuvo que contentarse con dar clases sustituyendo a su padre cuando estaba enfermo. Aún así, continuó sus investigaciones acerca de los invariantes por su cuenta.

Durante este periodo fue muy relevante el cambio de la dirección de Paul Gordan por la de \define{Ernst Fischer}{Ernst Fischer}\index{Ernst Fischer}, dada la reticencia de Gordan a los métodos de generalización de {Hilbert}{David Hilbert}\index{David Hilbert}, a los que denominó "Teología" con cierta sorna, lo que permitió a Noether disponer de más herramientas para sus investigaciones \citep{Weyl}.

Es, en este periodo es cuando empieza a tener renombre gracias a sus publicaciones.

Hilbert estaba interesado en los conocimientos y habilitades de Noether en el cálculo de invariantes para aplicarlo a sus trabajos sobre la \define{Teoría de la Relatividad General}{Teoría de la Relatividad General}\index{Teoría de la Relatividad General},
por lo que él y Felix Klein, invitaron a Noether en 1915 a trabajar con ellos en la Universidad de Gotinga. 
Aquí, una vez más, Noether sufrió la discriminación al ser privada de un puesto remunerado como profesora (conocido como \textit{privatdozent}) por ser mujer.
Aunque Hilbert intentó en varias ocasiones revertir esta situación ante el claustro de la Facultad, no tuvo éxito y en una de ellas espetó: "No veo que el sexo de la candidata sea un argumento contra su admisión como privatdozent. Después de todo, somos una universidad y no una casa de baños" (Carrasco 2004, 332).
Tras el final de la \define{Primera Guerra Mundial}{Primera Guerra Mundial}\index{Primera Guerra Mundial}, estalló en Alemánia la \define{República de Weymar}{República de Weymar}\index{República de Weymar}, lo que supuso una paulatina mejora en las condiciones de acceso de las mujeres a las universidades (además del sufragio femenino).
Así, Noether consigió ser reconocida oficialmente como profesora, aunque le costaría aun tres años que este reconocimiento conllevase una retribución económica \cite[333]{Carrasco}.












A pesar de la importancia de sus trabajos, Emmy Noether no tendría el reconocimiento academico que se merecía. Como comenta Einstein en una carta a {Hilbert}{David Hilbert}\index{David Hilbert} datada del 24 de Mayo de 1918:
""
“Gestern erhielt ich von Fr. Noether eine sehr interessante Arbeit ueber Invariantenbildung. Es imponiert mir, dass man diese Dinge von so allgemeinem Standpunkt u ¨ bersehen kann. Es h¨ atte den Goettinger Felgrauen nichts geschadet, wenn sie zu Frl. Noether in die Schule geschickt worden waeren. Sie scheint ihr Handwerk zu verstehen!”
Who Gave you the Epsilon? : And Other Tales of Mathematical History, edited by Marlow Anderson, et al., American Mathematical Society, 2009. ProQuest Ebook Central,
p.350




"Su significante y altruista trabajo, realizado en un periodo de varios años, fue recompensado por los nuevos governantes de Alemania con un despido, que le costaría a ella la pérdido de los medios para mantener su vida sencilla y la oportunidad de continuar con el estudios matemáticos."
Her unselfish, significant work over a period of many years was rewarded by new rulers of Germany with a dismissal, which cost her the means of maintaining her simple life and the opportunity to carry on her mathematical studies.
(Einstein 1935)





Memorial address: Emmy Noether by Hermann Weyl 1935