\chapter{Conclusiones}\label{ch:conclusiones}
Hemos definido las simetrías y hemos visto como podemos encontrarlas en la naturaleza, incluso en las propias ecuaciones del movimiento.

Hemos desarrollado el teorema de Noether, para entender como las simetrías tienen consecuencias muy profundas en las leyes de la física, como por ejemplo la conservación de la energía, al comprobar la relación entre simetrías y cantidades que se conservan.

Luego hemos visto ejemplos de simetrías en física clásica, y sus cantidades conservadas, para volver a realizar las mismas observaciones en física cuántica.

Gran parte de los avances en física de partículas, y los desarrollos de modelos teóricos como la teoría cuántica de campos, la cromodinámica cuántica, o incluso el modelo estándar de partículas usan las simetrías como argumento para validarlas.
Incluso cuando estas simetrías se rompen, se busca la forma de volver a tenerla, teniendo como consecuencia el añadido de un ente matemático que cumple con las condiciones para ser comparado con alguna partícula fundamental, el caso más notable de los últimos tiempos es la conjetura y descubrimiento del bosón de Higss.

El mecanismo de estudio parte de una situación física que se modeliza matemáticamente a través del concepto del Lagrangiano o del Hamiltoniano del sistema, y se estudian las transformaciones que dejan invariante la acción.
Estas transformaciones invariantes (simetrías) son las que hemos visto a lo largo de este trabajo.

¿Es la simetría una característica inevitable de la naturaleza?
Parece que sí, pero algunas de las últimas investigaciones, arrojan resultados que mantienen aún la duda.

La física cuántica no deja de sorprender ya incluso antes de su existencia, y como el niño rebelde de la física, sigue empeñada en decirnos que ni nuestras teorías ni nuestras experiencias están del todo desarrollados, y actualmente, ya se duda de que una teoría más completa debe considerar la ruptura de estas simetrías, incluso en conceptos tan arraigados en nosotros como es el tiempo.

Un ejemplo son las simulaciones sobre \define{cristales de tiempo}{cristales de tiempo} realizadas recientemente en computadoras cuánticas, que podrían ser el preludio del descubrimiento de asimetrías en el tiempo. También vemos aquí, la relevancia que puede tener este tipo de máquinas para la investigación de las propiedades de la mecánica cuántica, tal como predijo en su día Feynman.
https://www.quantamagazine.org/first-time-crystal-built-using-googles-quantum-computer-20210730/

Existen trabajos que dudan incluso de la simetría CPT, llevando asi la física a un nuevo nivel, pues tendríamos que cuestionarnos incluso si nuestros modelos sobre el origen del universo son correctos.

Por otro lado, existen teorías que van justo en la dirección contraría, es decir, en forzar la simetría al máximo, en lo que se conoce como teorías de supersimetría.
Aunque estas teorías están lejos de obtener confirmación experimental, prometen resolver problemas que actualmente no son explicables como la materia oscura.

Como conclusión, podemos afirmar que la simetría condiciona al modelo teórico, hasta tal punto que exigir dicha simetría permite conjeturar propiedades y existencia de partículas aún no descubiertas.
Tanto si dicha simetría se debe respetar como si no, el estudio de la simetría es fundamental para entender cual es el alcance del marco teórica de un modelo físico.