\chapter{Conclusiones}\label{ch:conclusiones}
Hemos definido las simetrías y hemos visto como podemos encontrarlas en la naturaleza.

Gracias al teorema de Noether, tenemos una potente herramienta que nos permite descubrir más acerca de las leyes de la física, cada vez que encontremos una simetría en la naturaleza.

¿Es la simetría una característica inevitable de la naturaleza?
Parece que sí, pero algunas de las últimas investigaciones, arrojan resultados que mantienen aún la duda.

Un ejemplo son las simulaciones sobre \define{cristales de tiempo}{cristales de tiempo} realizadas recientemente en computadoras cuánticas, que podrían ser el preludio del descubrimiento de asimetrías en el tiempo. También vemos aquí, la relevancia que puede tener este tipo de máquinas para la investigación de las propiedades de la mecánica cuántica, tal como predijo en su día Feynman.
https://www.quantamagazine.org/first-time-crystal-built-using-googles-quantum-computer-20210730/

