\section{Simetrías y cantidades conservadas}

Las leyes de conservación son fundamentales para entender los procesos físicos y nos permiten anticipar que situaciones pueden darse y cuales no.

Existen leyes de conservación globales, que se aplican a cualquier rama de la física como las leyes de conservación de la energía, el movimiento lineal, el movimiento angular y la carga eléctrica.

Otras son leyes de conservación locales, que sólo se pueden aplicar en contexto muy concretos, como las leyes de conservación de la paridad, el número leptónico, el número barionico, el isospin.

Cuando el lagrangiano de un sistema tiene una simetría $\sigma$, éste se mantiene invariante tras aplicar la simetría, y por tanto, toda trayectoria $\alpha$ es equivalente a $\sigma(\alpha)$, y estamos en condiciones de aplicar el teorema de Noether~\eqref{thm:noether} y afirmar por tanto que el flujo de Noether de $\alpha$ a $\sigma(\alpha)$ es constante.

\section{Homogeneidad del espacio y conservación del momento lineal}
Es lógico pensar que las ecuaciones de la física no varien según el lugar del espacio elegido, por eso, un cambio de coordenadas del tipo $\beta=\alpha+p_0$ donde $p_0\in\R^3$ debe dejar invariante el lagrangiano y dejar el flujo de Noether constante, es decir
\begin{equation}
	\label{eq:conservacion_momento_lineal}
	\frac{\partial L}{\partial\dot{\alpha}}(\beta-\alpha) = \frac{\partial L}{\partial\dot{\alpha}}p_0=\text{cte}\so\frac{\partial L}{\partial\dot{\alpha}}=\text{cte}
\end{equation}

\section{Isotropía del espacio y conservación del momento angular}
Es lógico pensar que las ecuaciones de la física no varien según la orientación elegida, por eso, una rotación en un plano del tipo $\beta=\alpha + (\alpha\times s)$ donde $s$ es un vector sobre el eje de rotación, debe dejar invariante el lagrangiano y dejar el flujo de Noether constante, es decir
\begin{equation}
	\label{eq:conservacion_momento_angular}
	\frac{\partial L}{\partial\dot{\alpha}}(\beta-\alpha) = \frac{\partial L}{\partial\dot{\alpha}}(\alpha\times s) = s \frac{\partial L}{\partial\dot{\alpha}}\times \alpha=\text{cte}\so\frac{\partial L}{\partial\dot{\alpha}}\times \alpha=\text{cte}
\end{equation}

\section{Homogeneidad del tiempo y conservación de la energía}
Es lógico pensar que las leyes de la física no varian con el tiempo, esto nos obliga a exigir que el lagrangiano de un sistema sea el mismo con respecto al tiempo, así pues, debemos tener que $\dot{L}=0$.
Aplicando la regla de la cadena tenemos:
\begin{align*}
	\dot{L}= & \frac{dL(\alpha, \dot{\alpha})}{dt}=\frac{\partial L}{\partial \alpha}\frac{d\alpha}{dt}+\frac{\partial L}{\partial \dot{\alpha}}\frac{d\dot{\alpha}}{dt} = \\
	\by{\ref{eq:euler-lagrange}} & \frac{d}{dt}\left(\frac{\partial L}{\partial \dot{\alpha}}\right)\dot{\alpha}+\frac{\partial L}{\partial \dot{\alpha}}\frac{d\dot{\alpha}}{dt}=\frac{d}{dt}\left(\frac{\partial L}{\partial \dot{\alpha}}\dot{\alpha}\right)\by{\ref{eq:energia_lagrangiana}}\frac{d}{dt}\left(E+L\right)=\dot{E}+\dot{L}
\end{align*}
La igualdad anterior nos lleva a la expresión $\dot{E}=0$, es decir, la energía no varía con el tiempo.
