\section{Simetrías en física clásica}\label{sec:simetrías-y-cantidades-conservadas-en-física-clásica}

\subsection{Simetrías y cantidades conservadas}\label{subsec:simetrías-y-cantidades-conservadas}
Las leyes de conservación son fundamentales para entender los procesos físicos y nos permiten comprender como evolucionan los sistemas.

Cuando el lagrangiano de un sistema presenta simetría en una transformación $\sigma$, éste se mantiene invariante tras aplicar dicha transformación, y por tanto, estamos en condiciones de aplicar el teorema de Noether~\eqref{thm:noether} y afirmar que la carga de Noether de $\alpha$ a $\sigma(\alpha)$ es constante.

\subsection{Homogeneidad del espacio y conservación del momento lineal}\label{subsec:homogeneidad-del-espacio-y-conservación-del-momento-lineal}
Es lógico pensar que las ecuaciones de la física no cambien según el lugar del espacio elegido, por eso, un cambio de coordenadas por traslación es una simetría.
¿Cual es la carga de Noether?

La simetría de traslación la podemos caracterizar por $f(q)=q+q_0$ para todo $q, q_0\in\R^n$ donde $q_0$ es constante.
Por el teorema de Noether~\eqref{thm:noether}, la carga de Noether es
\begin{equation}
	\label{eq:conservacion_momento_lineal}
	\frac{\partial \lagrangiano}{\partial\dot{\alpha}}(f(\alpha)-\alpha) = \frac{\partial \lagrangiano}{\partial\dot{\alpha}}q_0=\text{cte}\so\frac{\partial \lagrangiano}{\partial\dot{\alpha}}=\text{cte}
\end{equation}

Por tanto, podemos asegurar, que si en nuestro sistema físico no existe un observador privilegiado (homogeneidad del espacio), entonces el momento lineal se conserva.

\subsection{Isotropía del espacio y conservación del momento angular}\label{subsec:isotropic-del-espacio-y-conservación-del-momento-angular}
Es lógico pensar que las ecuaciones de la física no cambien según la orientación elegida, por eso, un cambio de coordenadas por rotación es una simetría.
¿Cual es la carga de Noether?

La simetría de rotación la podemos caracterizar por $f(q)=q+(q\times s)$ para todo $q\in\R^n$ y $s$ es un vector unitario sobre el eje de rotación.
Por el teorema de Noether~\eqref{thm:noether}, la carga de Noether es
\begin{equation}
	\label{eq:conservacion_momento_angular}
	\frac{\partial \lagrangiano}{\partial\dot{\alpha}}(f(\alpha)-\alpha) = \frac{\partial \lagrangiano}{\partial\dot{\alpha}}(\alpha\times s) = s \frac{\partial \lagrangiano}{\partial\dot{\alpha}}\times \alpha=\text{cte}\so\frac{\partial \lagrangiano}{\partial\dot{\alpha}}\times \alpha=\text{cte}
\end{equation}

Por tanto, podemos asegurar, que si en nuestro sistema físico no existe una dirección privilegiada (isotropía del espacio), entonces el momento angular se conserva.

\subsection{Homogeneidad del tiempo y conservación de la energía}\label{subsec:homogeneidad-del-tiempo-y-conservación-de-la-energía}
Es lógico pensar que las leyes de la física no cambien con el tiempo, por eso, el lagrangiano de un sistema no puede depender explícitamente del tiempo, así pues, debemos tener que $\frac{d\lagrangiano}{dt}=0$.
Por lo tanto tenemos:
\begin{align*}
	0 & = \frac{d\lagrangiano}{dt}=\frac{\partial \lagrangiano}{\partial \alpha}\frac{d\alpha}{dt}+\frac{\partial \lagrangiano}{\partial \dot{\alpha}}\frac{d\dot{\alpha}}{dt} ~\by{\ref{eq:euler-lagrange}} \frac{d}{dt}\left(\frac{\partial \lagrangiano}{\partial \dot{\alpha}}\right)\dot{\alpha}+\frac{\partial \lagrangiano}{\partial \dot{\alpha}}\frac{d\dot{\alpha}}{dt}=\\
	& = \frac{d}{dt}\left(\frac{\partial \lagrangiano}{\partial \dot{\alpha}}\dot{\alpha}\right)\by{\ref{eq:energía-lagrangiana}}\frac{d\left(E+\lagrangiano\right)}{dt}=\frac{d E}{dt}+\frac{d\lagrangiano}{dt}=\frac{d E}{dt}
\end{align*}
Tenemos que la energía no varía con el tiempo.
