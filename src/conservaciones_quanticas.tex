
\section{Simetrías y cantidades conservadas quánticas}\label{sec:simetrias-y-cantidades-conservadas-quanticas}
En la física quántica, también encontramos cantidades que se conservan por simetrías, no vamos a dar muchos detalles de estas relaciones por no extendernos en los conceptos y notaciones necearias.

En física de partículas, el isospín es un número cuántico relacionado con la interacción fuerte y aplicado a las interacciones del neutrón y el protón.
Las interación fuerte es invariante bajo rotación espacial, y aplicando el teorema de Noether nos lleva a la conservación del isospin de la partícula.

Como aplicación de este resultado, sabemos por tanto que como la interacción débil no conserva el isospin, entonces no puede ser invariante bajo rotación, y esto significa que esta interacción rompe una de las simetrías consideradas como esenciales hasta este momento, la simetría CPT

Carga <-> Transformación Gauge