
\chapter{Simetrías y cantidades conservadas en física quántica}\label{ch:simetrias-y-cantidades-conservadas-en-fisica-quantica}

No podemos presentar con el mismo detalle que en el capítulo anterior, la relación entre simetrías y cantidades conservadas en mecánica cuántica, pues el marco matemático es completamente diferente, y tendríamos que desarrollar y explicar conceptos como los operadores, generadores de simetrías, campos, variedades diferenciales, etc. que no es el objetivo de este trabajo.

Sin embargo, y por la importancia que tiene también en física clásica, sí le daremos más protagonismo a un caso particular: el electromagnetismo.

\section{Simetría en electromagnetismo}\label{sec:simetria-en-electromagnetismo}

Se dice que una teoria es una \define{teoría gauge}{Teoría Gauge} si es una teoría de campos donde el lagrangiano es invariante ante transformaciones locales sobre ciertas familias de operaciones diferenciables\autocite{MAQFT}.

Cuando un conjunto de simetrías de una teoría gauge tiene estructura de grupo, se dice que hay un \define{grupo de simetrías}{Grupo de simetrías}.
Y en general, las teorías gauge se puede expresar (caracterizar) como una \textquotedblleft combinación \textquotedblright especial de grupos de simetrías.

La teoría gauge más famosa es sin duda la teoría electromagnética de Maxwell, aquí tenemos una simetría que genera el grupo de simetrías $U(1)$ y que al usar el teorema de Noether sobre la invarianza del lagrangiano para esta simetría, obtenemos la \textbf{conservación de la carga eléctrica}.

\section{Otras simetrías}\label{sec:otras-simetrias}

En física quántica, una \define{carga}{Carga} es un generador arbitrario de una simetría contínua del sistema físico.
Cuando un sistema físico exhibe alguna simetría, el teorema de Noether implica la existencia de una corriente conservada.
Lo que \textquotedblleft fluye\textquotedblright\ en la corriente es la \textquotedblleft carga\textquotedblright, que es el generador del grupo de simetría (local).
Este carga a veces se denomina carga de Noether\autocite{W-CARGA}.

La teoría de gauge mas exitosa en la actualiadad es el \define{modelo estándar}{Modelo estándar}, que podemos caracterizar a través de sus grupos de simetrías $U(1)\times SU(2)\times SU(3)$.

\begin{itemize}
	\item Como ya hemos visto, la simetría en $U(1)$ conlleva a la conservación de la carga eléctrica.
	\item La simetría de $SU(2)$ conlleva a la conservación del isospín débil.
	\item La simetría de $SU(3)$ conlleva a la conservación de la carga de color.
\end{itemize}
