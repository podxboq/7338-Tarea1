\section{Introducción}\label{sec:introduccion}

Todo el mundo puede imaginar a qué nos referimos cuando hablamos de una simetría.
Consiste en una forma geométrica tal que, tras realizar alguna transformación, ésta queda igual.
Las simetrías están presentes en todas las actividades humanas, arquitectura, dibujo, música, etc.
Incluso en la naturaleza observamos simetrías con bastante frecuencia, la forma de una mariposa, las hojas de los arboles, las olas del mar, etc.

Pero las simetrías son algo más que bonitas formas geométricas, cuando se estudia con detalle los procesos físicos que ocurren en la naturaleza, se observa que la simetría es una propiedad que surge de forma natural.

Toda sistema físico puede ser representado mediante una modelo matemático que incorpora en sus ecuaciones los fundamentos del sistema real, y por tanto, podemos estudiar las simetrías desde un punto de vista matemático, pudiendo despues, interpretar los resultados matemáticos a realidades físicas.

El objetivo de este trabajo es presentar el formalismo matemático que permite, de las ecuaciones que modelan un sistema físico, estudiar las simetrias y obtener el significado físico.
Veremos que este significado físico, condiciona  muy profundas sobre ?



