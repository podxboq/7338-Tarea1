\section{Introducción}\label{sec:introduccion}

Cualquier persona puede imaginarse a qué nos referimos si hablamos de una simetría. Por ejemplo, la forma de una mariposa es simétrica, también las hojas de los árboles, las olas del mar, etc. Podemos observar simetrías en la naturaleza y entender, que en cierto modo, la naturaleza es propicia a las mismas.

Si lo definimos formalmente, podríamos decir que, la simetría es una característica consistente en realizar alguna transformación a un objeto y obtener el objeto original desde una perspectiva matemática. Esto viene dado porque a pesar de estas transformaciones, existe una cualidad que no varía.

Las transformaciones que describen simetrías físicas forman un grupo matemático, y como veremos, el estudio de estos grupos nos proporciona una herramienta muy potente para desarrollar modelos teóricos físicos.

Existen dos tipos de simetría:
\begin{itemize}
    \item Simetrías discretas. Cuando la cantidad de transformaciones que generan simetría en el sistema es un número entero determinado. Por ejemplo: en los ejemplos descritos, la simetría de una hoja, se da rotando sobre cierto eje 180º.
    \item Simetrías continuas. Cuando hay infinita cantidad de transformaciones posibles que generen simetría, dado esto por el carácter infinitesimal del sistema. Por ejemplo: si consideramos la rotación de la tierra respecto a su eje, cada pequeño movimiento infinitesimal del planeta genera una simetría.
\end{itemize}

Quizás en un primer momento, nos puede parecer que las simetrías son sólo bellas formas geométricas, pero, si estudiamos con detalle los procesos físicos, encontraremos que es una cualidad que surge de forma natural y con unas propiedades muy interesantes y útiles desde el punto de vista de la Física Teórica.

Estas propiedades llamaron la atención de una brillante matemática a principios del siglo XX llamada \define{Amalie Emmy Noether}{Emmy Noether} que, mediante la aplicación a las simetrías de unos modelos matemáticos muy conocidos en la Física Teórica y sus novedosos y revolucionarios conceptos sobre el Álgebra, llegó a conclusiones de gran transcendencia para el estudio de los fenómenos naturales.

La parte más importante de este trabajo es entender el llamado \define{Teorema de Noether}{Teorema de Noether}, que nos permitirá trasladar simetrías en las leyes de la física a conservaciones de valores fundamentales, para ello primero debemos explicar las herramientas en las que se apoya.

Comenzaremos el trabajo con un capítulo que explica la mecánica lagrangiana, necesaria para trabajar con los conceptos que se utilizan en el estudio de las simetrías. Dentro del mismo, hablaremos de las ecuaciones de Euler-Lagrange, el lagrangiano de un sistema y el principio de mínima acción.

Tras esto, pasaremos a situar la figura de Noether históricamente y a describir la parte de su importante obra, que está relacionada con las simetrías.

A continuación, utilizaremos los hallazgos de Noether para estudiar las simetrías en la mecánica clásica primero, y en la mecánica cuántica después.

Por último, debatiremos y expondremos las conclusiones del trabajo.