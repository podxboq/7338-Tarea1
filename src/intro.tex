\section{Introducción}\label{sec:introduccion}

Cualquier persona puede imaginarse a qué nos referimos si hablamos de una simetría. Por ejemplo, la forma de una mariposa es simétrica, también las hojas de los árboles, las olas del mar, etc. Así, podemos observar simetrías en la naturaleza con bastante frecuencia.
Si lo defiminos formalmente, podríamos decir que, una simetría es una forma gométrica tal que, tras realizar alguna transformación, la forma original y la transformada son equivalentes.

Quizás en un primer momento, nos puede parecer que las simetrías son sólo bellas formas geométricas, pero, si estudiamos con detalle los procesos físicos que ocurren en la naturaleza, podemos encontrar que la simetría, es una cualidad que surge de forma natural y con unas propiedades muy interesantes y útiles desde el punto de vista de la Física Teórica.

Estás propidades llamaron la atención de una brillante matemática a principios del siglo XX llamada Amalie Emmy Noether que, mediante la aplicación de unos modelos matemáticos muy conocidos en la Física Teórica a las simetrías, llegó a conclusiones muy de gran transcendencia para el estudio de los fenomenos naturales.

Expondremos así en este texto el llamado Torema de Noether y veremos las repercusiones que conlleva, pero primero debemos explicar las herramientas matemáticas en las que se apoya. 




