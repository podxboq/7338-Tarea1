\section{Introducción}\label{sec:introduccion}

Cualquier persona puede imaginarse a qué nos referimos si hablamos de una simetría. Por ejemplo, la forma de una mariposa es simétrica, también las hojas de los árboles, las olas del mar, etc. Así, podemos observar simetrías en la naturaleza con bastante frecuencia.
Si lo defiminos formalmente, podríamos decir que, la simetría es una característica que consiste en que, tras realizar alguna transformación a un objeto de estudio, el objeto original y el transformado son equivalentes. Esto viene dado porque a pasar de estas transformaciones, existe una cualidad que no varía.
Existen así dos tipos de simetría:
\begin{itemize}
    \item Simetrías discretas. Cuando la cantidad de transformaciones que generan simetría en el sistema es un número entero determinado. Por ejemplo: en los ejemplos descritos, las tranformaciones que se pueden hacer para que el sistema sea equivalente es sólo un, rotando sobre un eje.
    \item Simetrías contínuas. Cuando hay infinita cantidad de transformaciones posibles que generen simetria, dado esto por el caracter infinitesimal del sistema. Por ejemplo: si consideramos la rotación de la tierra respecto a su eje, cada pequeño movimiento infinitesimal del planeta genera una simetría.
\end{itemize}

Quizás en un primer momento, nos puede parecer que las simetrías son sólo bellas formas geométricas, pero, si estudiamos con detalle los procesos físicos que ocurren en la naturaleza, podemos encontrar que la simetría, es una cualidad que surge de forma natural y con unas propiedades muy interesantes y útiles desde el punto de vista de la Física Teórica.

Estás propidades llamaron la atención de una brillante matemática a principios del siglo XX llamada \define{Amalie Emmy Noether}{Emmy Noether} \index{Emmy Noether} que, mediante la aplicación a las simetrías de unos modelos matemáticos muy conocidos en la Física Teórica y sus novedosos y revolucionarios conceptos sobre el Álgebra, llegó a conclusiones de gran transcendencia para el estudio de los fenomenos naturales.

Expondremos así en este artículo el llamado \define{Teorema de Noether}{Teorema de Noether} \index{Teorema de Noether} y veremos las repercusiones que conlleva, pero primero debemos explicar las herramientas en las que se apoya. 




