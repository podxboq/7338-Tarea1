%! Author = podxboq
%! Date = 12/09/2022


% Preamble
\documentclass[12pt]{report}

% Packages
\usepackage[T1]{fontenc}
\usepackage[spanish]{babel}
\usepackage{csquotes}
\usepackage{amsmath}
\usepackage{amsthm}
\usepackage{amssymb}
\usepackage{fontspec}
\usepackage{imakeidx}
\usepackage{tikz}
\usepackage{float}
\usepackage[style=chicago-authordate]{biblatex}
\bibliography{main}
%\usepackage{cite}
\newtheorem{theorem}{Teorema}
\newtheorem{proposition}{Proposición}
\newtheorem{definition}{Definición}
\providecommand{\tq}{\mid}
\providecommand{\N}{\mathbb{N}}
\providecommand{\Z}{\mathbb{Z}}
\providecommand{\Q}{\mathbb{Q}}
\providecommand{\R}{\mathbb{R}}
\providecommand{\C}{\mathbb{C}}
\providecommand{\H}{\mathcal{H}}
\providecommand{\Im}[1]{Im(#1)}
\providecommand{\Re}[1]{Re(#1)}
\providecommand{\conjugate}[1]{\bar{#1}}
\providecommand{\pescalar}[2]{\langle #1,#2 \rangle}
\providecommand{\braket}[2]{\left\langle#1\mid#2\right\rangle}
\providecommand{\bra}[1]{\left\langle#1\right\rvert}
\providecommand{\ket}[1]{\left\lvert#1\right\rangle}
\providecommand{\so}{\Rightarrow}
\providecommand{\by}[1]{\overset{\fbox{\tiny #1}}{=}}
\providecommand{\maps}[3]{#1:#2\longrightarrow #3}
\providecommand{\coma}{,\thinspace}
\providecommand{\pari}[2]{(#1,\thinspace #2)}
\providecommand{\indexdots}[3]{#1=#2,\ldots,#3}
\providecommand{\define}[2]{\textbf{#1}}\index{#2}
\providecommand{\avg}[1]{\left\langle#1\right\rangle}
\providecommand{\abs}[1]{\lvert#1\rvert}
\providecommand{\nor}[1]{\lVert#1\rVert}
\providecommand{\operatoravg}[3]{\left\langle#1|#2|#3\right\rangle}
\providecommand{\cinfinity}[1]{\mathscr{C}^\infty(#1)}
\providecommand{\glossarydef}[3]{\newglossaryentry{#1}{name={#2},description={#3}}\gls{#1}}
\providecommand{\vec}[1]{\overrightarrow{#1}}
\newcommand{\set}[1]{\left\{#1\right\}}
\newcommand{\where}{\mathrel{}\middle|\mathrel{}}
\renewcommand*{\hbar}{{\mkern-1mu\mathchar'26\mkern-8mu\mathrm{h}}}
\renewcommand{\c}{\mathrm{c}}
\newcommand{\h}{\mathrm{h}}
\newcommand{\amplitud}{\mathrm{A}}
\newcommand{\longonda}{\lambda}
\newcommand{\frecuencia}{\nu}
\newcommand{\fase}{\alpha}
\newcommand{\frecuenciaangular}{\omega}
\newcommand{\longondaangular}{\mathrm{K}}
\newcommand{\periodo}{\mathrm{T}}
\newcommand{\energia}{\mathrm{E}}
\newcommand{\fuconda}{\Psi}
\newcommand\ct[1]{\text{\rmfamily\upshape #1}}
\newcommand\lagrangiano{\mathcal{L}}
\setmainfont{Calibri}
\renewcommand{\baselinestretch}{1.5}
\setlength{\parskip}{\baselineskip}
% Document
\makeindex
\begin{document}
    \title{La importancia de las simetrías en física}
    \author{Francisco Costa y Francisco Vidal}
    \date{Septiembre 2022}
    \maketitle
    \tableofcontents

    \section{Introducción}\label{sec:introduccion}

Cualquier persona puede imaginarse a qué nos referimos si hablamos de una simetría. Por ejemplo, la forma de una mariposa es simétrica, también las hojas de los árboles, las olas del mar, etc. Podemos observar simetrías en la naturaleza y entender, que en cierto modo, la naturaleza es propicia a las mismas.

Si lo definimos formalmente, podríamos decir que, la simetría es una característica consistente en realizar alguna transformación a un objeto y obtener el objeto original desde una perspectiva matemática. Esto viene dado porque a pesar de estas transformaciones, existe una cualidad que no varía.

Las transformaciones que describen simetrías físicas forman un grupo matemático, y como veremos, el estudio de estos grupos nos proporciona una herramienta muy potente para desarrollar modelos teóricos físicos.

Existen dos tipos de simetría:
\begin{itemize}
    \item Simetrías discretas. Cuando la cantidad de transformaciones que generan simetría en el sistema es un número entero determinado. Por ejemplo: en los ejemplos descritos, la simetría de una hoja, se da rotando sobre cierto eje 180º.
    \item Simetrías continuas. Cuando hay infinita cantidad de transformaciones posibles que generen simetría, dado esto por el carácter infinitesimal del sistema. Por ejemplo: si consideramos la rotación de la tierra respecto a su eje, cada pequeño movimiento infinitesimal del planeta genera una simetría.
\end{itemize}

Quizás en un primer momento, nos puede parecer que las simetrías son solo bellas formas geométricas, pero, si estudiamos con detalle los procesos físicos, encontraremos que es una cualidad que surge de forma natural y con unas propiedades muy interesantes y útiles desde el punto de vista de la Física Teórica.

Estas propiedades llamaron la atención de una brillante matemática a principios del siglo XX llamada \define{Amalie Emmy Noether}{Emmy Noether} que, mediante la aplicación de las simetrías de unos modelos matemáticos muy conocidos en la Física Teórica y sus novedosos y revolucionarios conceptos sobre el Álgebra, llegó a conclusiones de gran transcendencia para el estudio de los fenómenos naturales.

La parte más importante de este trabajo es entender el llamado \define{Teorema de Noether}{Teorema de Noether}, que nos permitirá trasladar simetrías en las ecuaciones a conservaciones de valores fundamentales, para ello primero debemos explicar las herramientas en las que se apoya.

Comenzaremos el trabajo con un capítulo que explica la mecánica lagrangiana, necesaria para trabajar con los conceptos que se utilizan en el estudio de las simetrías. Dentro del mismo, hablaremos de las ecuaciones de Euler-Lagrange, el lagrangiano de un sistema y el principio de mínima acción.

Tras esto, pasaremos a situar la figura de Noether históricamente y a describir la parte más importante de su obra.

A continuación, utilizaremos los hallazgos de Noether para estudiar las simetrías en la mecánica clásica primero, y en la mecánica cuántica después.

Por último, debatiremos y expondremos las conclusiones del trabajo.
    \section{Mecánica Lagrangiana}
Para poder entender el \define{Teorema de Noether}{Teorema de Noether} \index{Teorema de Noether}, piedra angular en el estudio de las simetrías, es preciso conocer primero un concepto propio de la Física denominado "Mecánica Lagrangiana".

A partir de la formulación de las leyes del movimiento de Newton, los sistemas físicos se estudiaban mediante la posición de los elementos que los componen (valores de x,y,z en el caso de coordenadas cartesianas; ángulos y módulos en el caso de coordenadas polares; etc) y sus interacciones. Sin embargo, esto suponía que el estudio de los sitemas se volviera demasiado complejo en determinadas situaciones.

Así, en 1788, el matemático Joseph-Louis Lagrange, inventó una manera de describir la mecánica, basada en la energía total del sistema, que suponía una simplificación del estudio gran cantidad de situaciones. Este modelo se basa en la ecuación de Euler-Lagrange que explicamos a continuación.


\subsection{El lagrangiano del sistema}\label{sec:el-lagrangiano-del-sistema}

La mecánica lagrangiana es una reformulación de la mecánica Newtoniana, introducida por Joseph-Louis de Lagrange en 1788, donde para cada partícula del sistema se consideran los valores que definen su posición, dichos valores reciben el nombre de \define{coordenadas generalizadas}{Coordenadas generalizadas}~\autocite{GTP}.
Llamamos \define{velocidades generalizadas}{Velocidades generalizadas} a la derivada respecto del tiempo de las coordenadas generalizadas.

En un sistema de $N$ partículas, tendremos para cada partícula, $n$ coordenadas generalizadas, denotamos por $q_i,\ i=1\dots N$ al vector formado por las coordenadas generalizadas de cada partícula y por $q=(q_1,\dots,q_N)$ al vector de las coordenadas generalidas del sistema.
Observamos que la dimensión de $q$ es $nN$.

Se define el \define{lagrangiano}{Lagrangiano} $\lagrangiano$, como la aplicación escalar que se forma mediante la energía cinética ($T$) y la energía potencial ($V$) de tal manera que:
\begin{equation}
	\label{eq:lagrangiano_clasico}
	\lagrangiano(q,\dot{q},t)=T-V=\sum\limits_{i=1}^{N} \frac{1}{2}m\|\dot{q_i}\|^2-V(q_i,t)
\end{equation}

\section{integral de acción}\label{sec:integral-de-accion}

Un concepto muy importante en mecánica, asociado al Lagrangiano $\lagrangiano$, es el de \define{integral de acción}{Integral de acción} o simplemente \define{acción}{Acción}, que es un valor numérico calculado a través de la integral del lagrangiano sobre el tiempo entre dos instantes.
Si $\maps{\alpha}{I=[t_0, t_1]}{\R^3}$ es una función que representa el movimiento de una partícula, la acción se define como
\begin{equation}
	\label{eq:accion}
	S_\lagrangiano(\alpha) = \int_{I}\lagrangiano(\alpha(t), \dot{\alpha}(t), t)\ dt
\end{equation}

\begin{proposition}
	En los términos de la definición~\eqref{eq:accion}, el funcional $S_\lagrangiano$ es diferenciable en $\alpha$ y su valor es
	\begin{equation}
		\label{eq:accion_diferencial}
		dS_\lagrangiano(\alpha)(h)=\int_{I}\left( \frac{\partial \lagrangiano}{\partial\alpha}h+\frac{\partial \lagrangiano}{\partial\dot{\alpha}}\dot{h}\right) dt
	\end{equation}
\end{proposition}
\begin{proof}
	Por la definición de diferencial tenemos que para cualquier funcion $h$
	\begin{equation*}
		S_\lagrangiano(\alpha + h) -  S_\lagrangiano(\alpha) = dS_\lagrangiano(\alpha)(h) +\mathcal{O}(h^2)
	\end{equation*}
	Por otra parte y por la definición de la acción
	\begin{equation*}
		\begin{split}
			S_\lagrangiano(\alpha + h) -  S_\lagrangiano(\alpha) = & \int_{I}\lagrangiano(\alpha(t)+h(t), \dot{\alpha}(t)+\dot{h}(t), t)\ dt - \\
			- & \int_{I}\lagrangiano(\alpha(t), \dot{\alpha}(t), t)\ dt =\\
			=& \int_{I}\left( \frac{\partial \lagrangiano}{\partial\alpha}h+\frac{\partial \lagrangiano}{\partial\dot{\alpha}}\dot{h}\right)\ dt+\mathcal{O}(h^2)
		\end{split}
	\end{equation*}

	Donde en la última igualdad hemos usado el desarrollo de Taylor en una variable.
	Por lo tanto, por la unicidad de diferencial de funciones, tenemos que:

	\begin{equation*}
		dS_\lagrangiano(\alpha)(h) =\int_{I}\left( \frac{\partial \lagrangiano}{\partial\alpha}h+\frac{\partial \lagrangiano}{\partial\dot{\alpha}}\dot{h}\right) dt
	\end{equation*}

\end{proof}

\section{Ecuación de Euler-Lagrange}\label{sec:ecuacion-de-euler-lagrange}

En la mayoría de los casos, a la hora de estudiar un sistema físico, podemos centrarnos exclusivamente en las variaciones (modificaciones) que ocurren en el mismo. Medante el "\define{Cálculo Variacional}{Cálculo Variacional}\index{Cálculo Variacional}", que es un apartado de las matemáticas que se ocupa de estudiar un sistema funcional a través de los máximos y mínimos de sus funciones, podemos obtener la información necesaria para comprender los cambios de un sistema determinado.

El problema del llamado "Principio Variacional" surge en la historia de las matemáticas a través del estudio del problema de la "\define{Curva Braquistócrona}{Curva Braquistócrona}\index{Curva Braquistócrona}", planteado inicialmente por Johann Bernoulli en 1696 <<<FALTA CITA>>>.

Si consideramos un sistema sometido a fuerzas gravitaroria y en el que hay dos puntos a distinta altura, se llama Curva Braquistócrona, a aquella que hace que el tiempo que tarda un cuerpo inicialmente parado que está situado en el punto más alto (al que llamaremos A), en llegar al punto más bajo (B), sea mínimo.

Intuitivamente podría paracer que la función que cumple esto sería la línea recta, pero demostraremos a continuación que esto no es así.

A continuación vemos un esquema del problema:

DIBUJO EXPLICANDO LA CURVA BRAQUISTÓCRONA

Como sabemos, la velocidad a la que se mueve un objeto es igual al espacio que recore en un tiempo determinado. Por lo que, suponiendo una magnitudes funcionales (no escalares) el tiempo que tarda el objeto en caer del punto A al punto B es:
\begin{equation}
	\label{eq:tiempo}
	t_{A\xrightarrow{}B}=\int_{A}^{B}\frac{ds}{v}
\end{equation}

Sabemos que la velocidad del cuerpo a una determinada altura "y" es:
\begin{equation}
	\label{eq:velocidad_caida}
	v=\sqrt{2gy}
\end{equation}
Con "g" igual a la aceleración gravitatoria.


\begin{theorem}
	Una función real $\alpha$ alcanza un valor extremo (máximo o mínimo) en el funcional $S_L$ sí y sólo sí se da la siguiente igualdad llamada \define{ecuación de Euler-Lagrange}{Ecuación de Euler-Lagrange}
	\begin{equation}
		\label{eq:euler-lagrange}
		\frac{\partial \lagrangiano}{\partial \alpha}=\frac{d}{dt}\left( \frac{\partial \lagrangiano}{\partial \dot{\alpha}} \right)
	\end{equation}
\end{theorem}
\begin{proof}
	Una función alcanza un valor máximo o mínimo en el funcional $S_L$ cuando su diferencial se hace cero, es decir, que $dS_\lagrangiano(\alpha)(h)=0$ para cualquier función $h$.
	\begin{equation*}
		\begin{split}
			0 = dS_\lagrangiano(\alpha)(h)~\by{\ref{eq:accion_diferencial}}&\int_{I}\left( \frac{\partial \lagrangiano}{\partial\alpha}h+\frac{\partial \lagrangiano}{\partial\dot{\alpha}}\dot{h}\right) dt =\\
			= & \int_{I}\left( \frac{\partial \lagrangiano}{\partial\alpha}h+\frac{d}{dt}\left( \frac{\partial \lagrangiano}{\partial \dot{\alpha}}h \right)-\frac{d}{dt}\left( \frac{\partial \lagrangiano}{\partial \dot{\alpha}} \right)h\right)\ dt = \\
			= & \int_{I}\left( \frac{\partial \lagrangiano}{\partial \alpha}h-\frac{d}{dt}\left( \frac{\partial \lagrangiano}{\partial \dot{\alpha}} \right)h\right)\ dt +\int_{I}\frac{d}{dt}\left( \frac{\partial \lagrangiano}{\partial \dot{\alpha}}h \right)\ dt = \\
			= & \int_I\left( \frac{\partial \lagrangiano}{\partial \alpha}-\frac{d}{dt}\left( \frac{\partial \lagrangiano}{\partial \dot{\alpha}} \right)\right)h\ dt + \left[ \frac{\partial \lagrangiano}{\partial \dot{\alpha}}h \right]_{t_0}^{t_1}
		\end{split}
	\end{equation*}
	En particular si nos restringimos a las funciones $h$ que cumplen $h(t_0)=0=h(t_1)$, el segundo sumando vale $0$ y el para el primer sumando tenemos que
	\begin{equation*}
		\int_{I}\left( \frac{\partial \lagrangiano}{\partial \alpha}-\frac{d}{dt}\left( \frac{\partial \lagrangiano}{\partial \dot{\alpha}} \right)\right)h\ dt = 0 \so \frac{\partial \lagrangiano}{\partial \alpha}-\frac{d}{dt}\left( \frac{\partial \lagrangiano}{\partial \dot{\alpha}} \right) = 0
	\end{equation*}
\end{proof}

\section{La energía en función del lagrangiano}\label{sec:la-energia-en-funcion-del-lagrangiano}


\subsection{El lagrangiano del sistema}
Ahora que ya sabemos como resolver los problemas variacionales, podemos pasar a definir un sistema en función a las variacione sque consideremos relevantes en el mismo.
Consideremos los funcionales de las posiciones y velocidades que los elementos del sistema.

La mecánica lagrangiana es una reformulación de la mecánica Newtoniana, introducida por Joseph-Louis de Lagrange en 1788, donde para cada partícula del sistema se considera su posición y velocidad como variables, definiendo así un espacio de fases de 6N+1 dimensiones, donde N es el número de partículas y 6=3+3 son 3 dimensiones para la posición y 3 dimensiones para la velocidad y +1 para el tiempo.

Sobre este espacio de fases se define el lagrangiano ($L$), que es una aplicación escalar que se forma mediante la energía cinética ($T$) y la energía potencial ($V$) de tal manera que:
\begin{equation}
	\label{eq:lagrangiano_clasico}
	L(\vec{r_1},\dots,\vec{r_N},\dot{\vec{r_1}},\dots,\dot{\vec{r_N}},t)=T-V=\sum\limits_{i=1}^{i=N} \frac{1}{2}m\|\dot{\vec{r_i}}\|^2-V(\vec{r_i},t)
\end{equation}

\subsection{Acción de un funcional}
	Sea $\maps{L}{\R^n\times\R^n\times\R}{\R}$ una aplicación real y $\maps{\alpha}{I=[t_0, t_1]}{\R^n}$ una función real, definimos el funcional \define{acción de L}{acción} a:
	\begin{equation}
		\label{eq:accion}
		S_L(\alpha) = \int_{I}L(\alpha(t), \dot{\alpha}(t), t)\ dt
	\end{equation}

\begin{proposition}
	En los términos de la definición~\eqref{eq:accion}, el funcional $S_L$ es diferenciable en $\alpha$ y su valor es
	\begin{equation}
		\label{eq:accion_diferencial}
		dS_L(\alpha)(h)=\int_{I}\left( \frac{\partial L}{\partial\alpha}h+\frac{\partial L}{\partial\dot{\alpha}}\dot{h}\right) dt
	\end{equation}
\end{proposition}
\begin{proof}
	Por la definición de diferencial y realizando el desarrollo de Taylor tenemos que
	\begin{equation*}
		\begin{split}
			S_L(\alpha + h) -  S_L(\alpha)& = \int_{I}L(\alpha(t)+h(t), \dot{\alpha}(t)+\dot{h}(t), t)\ dt - \int_{I}L(\alpha(t), \dot{\alpha}(t), t)\ dt =\\
			&\by{\ref{eq:polinomio-taylor-dos-variables}}\int_{I}\left( \frac{\partial L}{\partial\alpha}h+\frac{\partial L}{\partial\dot{\alpha}}\dot{h}\right)\ dt+\mathcal{O}(h^2)\so\\
			dS_L(\alpha)(h) &=\int_{I}\left( \frac{\partial L}{\partial\alpha}h+\frac{\partial L}{\partial\dot{\alpha}}\dot{h}\right) dt
		\end{split}
	\end{equation*}
\end{proof}


\subsection{Principio de mínima acción}
Un desarrollo interesante, y que necesitaremos más adelante para entender la conservación de la energía, es expresar como depende la energía del sistema en función del lagrangiano, para ello, vamos a considera un sistema físico con una sóla partícula, pues la generación a $N$ partículas es directa, observemos que:
\begin{equation*}
	\frac{\partial\lagrangiano}{\partial\dot{q}} =m\dot{q}\so \frac{\partial\lagrangiano}{\partial\dot{q}}\dot{q} =m\dot{q}^2 = 2K
\end{equation*}

Por lo tanto:
\begin{equation}
	\label{eq:energia-lagrangiana}
	E=K+T=K+T+K-K=2K-\lagrangiano=\frac{\partial\lagrangiano}{\partial\dot{q}}\dot{q}-\lagrangiano
\end{equation}

\section{Principio de mínima acción}\label{sec:principio-de-minima-accion}

El principio de mínima acción o principio de Hamilton, es un postulado básico de la física para describir la evolución a lo largo del tiempo del estado de movimiento de una partícula.

El \textbf{principio de mínima acción} o \textbf{principio de Hamilton}, dice que las partículas se mueven a traves de trayectorias que minimizan la acción del lagrangiano.

    \section{Principio de mínima acción}\label{sec:principio-de-minima-accion}

Cuando intentamos predecir la trayectoria de un cuerpo en movimiento, en la mayoría de los casos lo veremos fácilmente de manera intuitiva. Sin embargo, esta no resulta tan obvia a la hora de trabajar en el papel. Por lo que necesitamos alguna formula que nos ayude a determinarla. Esta es el llamado \textbf{principio de mínima acción}.

El \define{principio de mínima acción}{Principio de mínima acción} o \define{principio de Hamilton}{Principio de Hamilton}, es un postulado que dice que el movimiento del sistema descrito por las coordenadas generalizadas es aquel que minimiza la acción del lagrangiano.

Es decir, si tenemos $\lagrangiano(q,\dot{q}, t)$ el lagrangiano del sistema, si $\alpha(t)$ es una curva que representa la trayectoria dentro del sistema entre dos instantes de tiempo $t_0$ y $t_1$, entonces:
\begin{equation*}
	dS_\lagrangiano(\alpha)=0\so \frac{\partial \lagrangiano}{\partial \alpha}=\frac{d}{dt}\left( \frac{\partial \lagrangiano}{\partial \dot{\alpha}} \right)
\end{equation*}

El \textbf{principio de mínima acción} o \textbf{principio de Hamilton} es una consecuencia de la segunda ley de Newton, que nos informa acerca de las trayectorias que sigen los cuerpos en un sistema inercial:
"Las leyes de Newton pueden ser enunciadas, no en la forma de F = m·a, si no en la siguiente forma: la energía cinética media menos la enegía potencial media es la mínima posible para la trayectoria de un objeto que va de un punto a otro" (\cite{Feynman}).

Como sabemos, el resultado de la energía cinética menos la energía potencial, conforma el lagrangiano del sistema como se comentó en \eqref{eq:lagrangiano_clasico}.
\begin{equation}
	\label{eq:lagrangiano_clasico_corto}
	\lagrangiano(q,\dot{q},t)=T-V
\end{equation}

Pero nos interesa precisamente el valor de esas energías entre dos puntos determinados, esto nos lo da la acción, tal y como se enunció anteriormente en \eqref{eq:accion}
\begin{equation}
	S_\lagrangiano(\alpha) = \int_{I}\lagrangiano(\alpha(t), \dot{\alpha}(t), t)\ dt
\end{equation}

Así, si minimizamos el valor de la acción, podemos obtener la función que nos da la trayectoria que seguirá el objeto sujeto a lagrangiano indicado.
\begin{equation}
    \delta(S) = \delta \int_{I}\lagrangiano(\alpha(t), \dot{\alpha}(t), t)\ dt
\end{equation}

Gracias al cálculo variacional, sabemos que es condición necesaria que la derivada de una función, sea cero donde la función sea mínima. Consideremos que queremos determinar el camino que seguirá un cuerpo desde el punto $x_1$ al punto $x_2$ y consideremos también la función $\eta$ que representa la direrencia entre cualquier curva aleatorio y el camino que se seguirá, con $\eta{x_1} = \eta{x_2} = 0$:
\begin{equation}
    \label{eq:accion1}
    \frac{dS}{d\alpha} = \int_{x_1}^{x_2}\frac{\partial\lagrangiano}{\partial\alpha}dx = \int_{x_1}^{x_2}(\eta\frac{\partial\lagrangiano}{\partial y} + \dot{\eta}\frac{\partial\lagrangiano}{\partial\dot{y}})dx = 0
\end{equation}

Integrando por partes:
\begin{equation}
    \int_{x_1}^{x_2}\dot{\eta}(x)\frac{\partial\lagrangiano}{\partial\dot{y}}dx = [\eta(x)\frac{\partial\lagrangiano}{\partial\dot{y}}]_{x_1}^{x_2} - \int_{x_1}^{x_2}\eta(x)\frac{d}{dx}(\frac{\partial\lagrangiano}{\partial\dot{y}})dx
\end{equation}

Recordemos que $\eta(x_1) = \eta(x_2) = 0$, el primer témino del segundo miembro es cero. Entonces:
\begin{equation}
    \int_{x_1}^{x_2}\dot{\eta}(x)\frac{\partial\lagrangiano}{\partial\dot{y}}dx = - \int_{x_1}^{x_2}\eta(x)\frac{d}{dx}(\frac{\partial\lagrangiano}{\partial\dot{y}})dx
\end{equation}

Susituyendo por \eqref{eq:accion1}:
\begin{equation}
    \int_{x_1}^{x_2}\eta(x)(\frac{\partial\lagrangiano}{\partial y} - \frac{d}{dx}\frac{\partial\lagrangiano}{\partial\dot{y}})dx = 0
\end{equation}

Como tiene que cumplirse esta ecuación para cualquier $\eta{x}$, entonces lo que hay entre paréntesis debe ser cero:
\begin{equation}
    \frac{\partial\lagrangiano}{\partial y} - \frac{d}{dx}\frac{\partial\lagrangiano}{\partial\dot{y}} = 0
\end{equation}

(\cite{Taylor})

Con lo que hemos llegado a la ecuaciones de Lagrange.

Resumiendo, el principio de mínima acción, establece que la trayectoria que seguirá un cuerpo en movimiento, será aquella, de todas las posibles, que minimice la acción.

Tenemos así, finalmente, tres formas de determinar la trayectoria de un cuerpo mediante la mecánica clásica (\cite[264]{Taylor}):
\begin{itemize}
    \item La segunda ley de Newton F = m \cdot a.
    \item Las ecuaciones de Lagrange.
    \item El Principio de Hamilton.
\end{itemize}
    \chapter{Emmy Noether}\label{ch:emmy-noether}

\section{Biografía}\label{sec:biografia}

No se puede hablar del estudio de las simetrías sin hacerlo de \define{Amalie Emmy Noether}{Emmy Noether}, nacida el 23 de Marzo de 1882 en Erlangen, Alemania.

Procedente de una familia de confesión Judía, el padre de Emmy, \define{Max Noether}{Max Noether}, fue un reconocido matemático autodidacta que consiguió un puesto en la Universidad de Erlangen.

A pesar de ello, Emmy no mostró especial interes en las matemáticas de joven.
Recibió, de su madre, una educación acorde a lo que le esperaba a una mujer en su tiempo (cocina y demás tareas del hogar) y también clases de clavecín y danza (sin mucho entusiasmo por su parte).
Le gustaba frecuentar los bailes que organizaban los amigos de su padre y tenía más interes en aprender lenguas.
Tanto es así, que se preparó y superó los exámente para ejercer de profesora de Inglés y Francés para los colegios de niñas en el estado de Baviera (\cite{Carrasco}).

Tras eso, quiso proseguir con sus estudios de lenguas en la universidad, sin embargo, a las mujeres no les estaba permitido matricularse de forma ordinaria en la universidad de Erlangen.
Para ir a las clases tendría que hacerlo de forma extraoficial y con el consentimiento de cada profesor, aunque sí que podría presentarse a los exámenes que certificaban su paso por la universidad (\cite{Kimberling}).
Durante sus estudios universitarios, decidió pasarse de la lengua a las matemáticas, y en 1903, se examina en Nuremberg bajo la dirección del matemático Aurel Vo\beta (\cite{Kimberling}).

En 1903, Emmy empieza sus clases en la Universidad de Gotinga, impartidas por eminentes matemáticos, entre otros, \define{David Hilbert}{David Hilbert} que se convertiría en su principal apoyo contra las trabas que, como mujer, tendría en el entorno académico.

Pero seis meses después, volvería a Erlangen, pues ya permitían matricularse a las mujeres (\cite{Kimberling}).

En 1907, presenta su tesis doctoral \textquote{Über die Bildung des Formensystems der ternären biquadratischen Form} (Sobre los sistemas completos de invariantes para formas ternarias bicuadráticas), bajo la supervisiónd de \define{Paul Gordan}{Paul Gordan}, matemático que se había especializado en invariantes.

En esa época, las mujeres tampoco tenían acceso a la posición de profesoras remuneradas en la universidad, por lo que Emmy tuvo que contentarse con dar clases sustituyendo a su padre cuando estaba enfermo.
Aún así, continuó sus investigaciones acerca de los invariantes por su cuenta.

Durante este periodo fue muy relevante el cambio de la dirección de Paul Gordan por la de \define{Ernst Fischer}{Ernst Fischer}, dada la reticencia de Gordan a los métodos de generalización de \define{Hilbert}{David Hilbert}, a los que denominó \textquotedblleft Teología\textquotedblright\ con cierta sorna, lo que permitió a Noether disponer de más herramientas para sus investigaciones (\cite{Weyl}).

En este periodo es cuando empieza a tener renombre gracias a sus publicaciones.

Hilbert estaba interesado en los conocimientos y habilidades de Noether en el cálculo de invariantes para aplicarlo a sus trabajos sobre la \define{Teoría de la Relatividad General}{Teoría de la Relatividad General}\index{Teoría de la Relatividad General},
por lo que él y Felix Klein, invitaron a Noether en 1915 a trabajar con ellos en la Universidad de Gotinga.
Aquí, una vez más, Noether sufrió la discriminación al ser privada de un puesto remunerado como profesora (conocido como \textit{privatdozent}) por ser mujer.
Aunque Hilbert intentó en varias ocasiones revertir esta situación ante el claustro de la Facultad, no tuvo éxito y en una de ellas espetó: \textquote{No veo que el sexo de la candidata sea un argumento contra su admisión como privatdozent. Después de todo, somos una universidad y no una casa de baños} (\cite[332]{Carrasco}).

A pesar de la importancia de sus trabajos, Emmy Noether no tendría el reconocimiento academico que se merecía.
Como comenta Einstein en una carta a \define{Hilbert}{David Hilbert}: \textquote{Ayer recibí un artículo muy interesante de la Noether sobre la formación de invariantes. Me impresiona que estas cosas se puedan pasar por alto. No habría perjudicado a los Goettinger Felgrauen si los hubieran enviado a estudiar con la señorita Noether. ¡Parece conocer su oficio!} (\cite{Kimberling}).

Tras el final de la \define{Primera Guerra Mundial}{Primera Guerra Mundial}, estalló en Alemánia la \define{República de Weymar}{República de Weymar}, lo que supuso una paulatina mejora en las condiciones de acceso de las mujeres a las universidades (además del sufragio femenino).
Así, Noether consigió ser reconocida oficialmente como profesora, aunque le costaría aun tres años que este reconocimiento conllevase una retribución económica (\cite[333]{Carrasco}).

Durante este periodo de tiempo, el trabajo de Noether fue clave para la formulación de la \define{Teoría de la Relatividad General}{Teoría de la Relatividad General}\index{Teoría de la Relatividad General} (\cite{Weyl}) y transcendería la fama de Noether en el mundo académico y obtuvo el reconocimiento que se merecía.
Tuvo alumnos que vinieron desde diversas partes del mundo para estudiar con ella, a los que se concía como los \textquotedblleft los chicos de Noether\textquotedblright\ (\cite{Carrasco}).

Sin embargo, años después, la cosas se complicarían.
La llegada de \define{Hitler}{Hitler}\index{Hitler} al poder, supuso la expulsión de las universidad de las personas judías y Noether tuvo que unirse a la larga lista de brillantes cientificos que abandonaron Alemania por la persecución Nazi (junto con Einstein, Von Neumann, etc), y que tomaron refugio en los \define{Estados Unidos de América}{Estados Unidos de América}\index{Estados Unidos de América} (\cite{Kimberling}).
Como diría Einstein: \textquote{Su significante y altruista trabajo, realizado en un periodo de varios años, fue recompensado por los nuevos governantes de Alemania con un despido, que le costaría a ella la pérdido de los medios para mantener su vida sencilla y la oportunidad de continuar con sus estudios matemáticos} (\cite{Einstein}).

Una vez en América, trabajó impartiendo clases de matemáticas en el \define{Bryn Mawr College}{Bryn Mawr College}, una universisdad femenina y cooperó con el \define{Instituto para Estudios Avanzados de Princeton}{Princeton}, donde trabajaba en ese momento Einstein (\cite{Carrasco}).

Dos años después, el 14 de Abril de 1935, con 54 años, Emmy fallecería a causa de complicaciones cardiacas en una operación.

\section{Flujo}\label{sec:flujo}

Para entender bien el primer teorema de Noether, y sobre todo, para poder entender la relación entre simetrías y conservación de cantidades, que se verán en los siguientes capítulos, es necesario presentar el concepto del flujo de Noether.

Tenemos un sistema físico, con un Lagrangiano $\lagrangiano$ y $\maps{\alpha,\beta}{I=[t_0, t_1]}{\R^3}$ dos funciones, tales que los lagrangianos para ambas funciones está relacionado por una tercera función real que cumpla $\lagrangiano(\beta, \dot{\beta}, t) = \lagrangiano(\alpha, \dot{\alpha}, t)+\dot{f}$.
Entonces las respectivas acciones están relacionadas por una constante, pues

\begin{equation}
	\label{eq:variacion-accion}
	\begin{split}
		S_\lagrangiano(\beta) & = \int_I \lagrangiano(\beta, \dot{\beta}, t) dt = \int_I \lagrangiano(\alpha, \dot{\alpha}, t)\ dt+\int_I \dot{f}dt=S_\lagrangiano(\alpha) +[f]_{t_0}^{t_1}
	\end{split}
\end{equation}

Y en este caso, la relación entre las acciones es

\begin{equation}
	\label{eq:variacion-accion-relacion}
	S_\lagrangiano(\beta) - S_\lagrangiano(\alpha) = f(t_1) - f(t_0)
\end{equation}

Como la diferencia entre las acciones es una constante, sus diferenciales son iguales, y por tanto si $\alpha$ es una trayectoria, también lo es $\beta$ y además tenemos que

\begin{equation}
	\label{eq:diferencial_trayectoria}
	\begin{split}
	[f]
		_{t_0}^{t_1} = & S_\lagrangiano(\beta) - S_\lagrangiano(\alpha) = S_\lagrangiano(\alpha + (\beta-\alpha))-S_\lagrangiano(\alpha)=dS_\lagrangiano(\alpha)(\beta-\alpha)=\\
		\by{\ref{eq:accion_diferencial}} & \int_{I}\left( \frac{\partial \lagrangiano}{\partial\alpha}(\beta-\alpha)+\frac{\partial \lagrangiano}{\partial\dot{\alpha}}(\dot{\beta}-\dot{\alpha})\right) dt = \\
		\by{\ref{eq:euler-lagrange}} & \int_{I}\left( \frac{d}{dt}\left(\frac{\partial \lagrangiano}{\partial\dot{\alpha}}\right)(\beta-\alpha)+\frac{\partial \lagrangiano}{\partial\dot{\alpha}}(\dot{\beta}-\dot{\alpha})\right) dt=\\
		= & \int_{I}\frac{d}{dt}\left(\frac{\partial \lagrangiano}{\partial\dot{\alpha}}(\beta-\alpha)\right) dt = \frac{\partial \lagrangiano}{\partial\dot{\alpha}}(\beta-\alpha) + \text{ cte }
	\end{split}
\end{equation}

Es decir, que para trayectorias cuyos lagrangianos se diferencia por la diferencial de una función, existe una constante $\ct{C}$ tal que
\begin{equation}
	\label{eq:flujo_noether_para_trayectorias}
	\frac{\partial \lagrangiano}{\partial\dot{\alpha}}(\beta-\alpha) = \ct{cte}
\end{equation}

\begin{definition}
	Dadas dos trayectorias $\maps{\alpha,\beta}{I}{\R^3}$, llamamos \define{flujo de Noether}{Flujo de Noether} de $\alpha$ a $\beta$ a
	\begin{equation}
		\label{eq:flujo_noether}
		\frac{\partial \lagrangiano}{\partial\dot{\alpha}}(\beta-\alpha)
	\end{equation}
\end{definition}

\section{Primer teorema de Noether}\label{sec:primer-teorema-de-noether}

Llegado a este punto, ya estamos en condiciones de enunciar el primer teorema de Noether, cuya demostración no sino el desarrollo de este capítulo hasta llegar a la expresión~\ref{eq:flujo_noether_para_trayectorias}.

\begin{theorem}
	\label{thm:noether}
	Si en un sistema con lagrangiano $\lagrangiano$ dos trayectorias $\alpha$ y $\beta$ cumple que $\lagrangiano(\beta, \dot{\beta}, t) = \lagrangiano(\alpha, \dot{\alpha}, t)+\dot{f}$ entonces el flujo de Noether de $\alpha$ a $\beta$, permanece constante.
\end{theorem}
    \section{Simetrías en física clásica}\label{sec:simetrías-y-cantidades-conservadas-en-física-clásica}

\subsection{Simetrías y cantidades conservadas}\label{subsec:simetrías-y-cantidades-conservadas}
Las leyes de conservación son fundamentales para entender los procesos físicos y nos permiten comprender como evolucionan los sistemas.

Cuando el lagrangiano de un sistema presenta simetría en una transformación $\sigma$, éste se mantiene invariante tras aplicar dicha transformación, y por tanto, estamos en condiciones de aplicar el teorema de Noether~\eqref{thm:noether} y afirmar que la carga de Noether de $\alpha$ a $\sigma(\alpha)$ es constante.

\subsection{Homogeneidad del espacio y conservación del momento lineal}\label{subsec:homogeneidad-del-espacio-y-conservación-del-momento-lineal}
Es lógico pensar que las ecuaciones de la física no cambien según el lugar del espacio elegido, por eso, un cambio de coordenadas por traslación es una simetría.
¿Cual es la carga de Noether?

La simetría de traslación la podemos caracterizar por $f(q)=q+q_0$ para todo $q, q_0\in\R^n$ donde $q_0$ es constante.
Por el teorema de Noether~\eqref{thm:noether}, la carga de Noether es
\begin{equation}
	\label{eq:conservacion_momento_lineal}
	\frac{\partial \lagrangiano}{\partial\dot{\alpha}}(f(\alpha)-\alpha) = \frac{\partial \lagrangiano}{\partial\dot{\alpha}}q_0=\text{cte}\so\frac{\partial \lagrangiano}{\partial\dot{\alpha}}=\text{cte}
\end{equation}

Por tanto, podemos asegurar, que si en nuestro sistema físico no existe un observador privilegiado (homogeneidad del espacio), entonces el momento lineal se conserva.

\subsection{Isotropía del espacio y conservación del momento angular}\label{subsec:isotropic-del-espacio-y-conservación-del-momento-angular}
Es lógico pensar que las ecuaciones de la física no cambien según la orientación elegida, por eso, un cambio de coordenadas por rotación es una simetría.
¿Cual es la carga de Noether?

La simetría de rotación la podemos caracterizar por $f(q)=q+(q\times s)$ para todo $q\in\R^n$ y $s$ es un vector unitario sobre el eje de rotación.
Por el teorema de Noether~\eqref{thm:noether}, la carga de Noether es
\begin{equation}
	\label{eq:conservacion_momento_angular}
	\frac{\partial \lagrangiano}{\partial\dot{\alpha}}(f(\alpha)-\alpha) = \frac{\partial \lagrangiano}{\partial\dot{\alpha}}(\alpha\times s) = s \frac{\partial \lagrangiano}{\partial\dot{\alpha}}\times \alpha=\text{cte}\so\frac{\partial \lagrangiano}{\partial\dot{\alpha}}\times \alpha=\text{cte}
\end{equation}

Por tanto, podemos asegurar, que si en nuestro sistema físico no existe una dirección privilegiada (isotropía del espacio), entonces el momento angular se conserva.

\subsection{Homogeneidad del tiempo y conservación de la energía}\label{subsec:homogeneidad-del-tiempo-y-conservación-de-la-energía}
Es lógico pensar que las leyes de la física no cambien con el tiempo, por eso, el lagrangiano de un sistema no puede depender explícitamente del tiempo, así pues, debemos tener que $\frac{d\lagrangiano}{dt}=0$.
Por lo tanto tenemos:
\begin{align*}
	0 & = \frac{d\lagrangiano}{dt}=\frac{\partial \lagrangiano}{\partial \alpha}\frac{d\alpha}{dt}+\frac{\partial \lagrangiano}{\partial \dot{\alpha}}\frac{d\dot{\alpha}}{dt} ~\by{\ref{eq:euler-lagrange}} \frac{d}{dt}\left(\frac{\partial \lagrangiano}{\partial \dot{\alpha}}\right)\dot{\alpha}+\frac{\partial \lagrangiano}{\partial \dot{\alpha}}\frac{d\dot{\alpha}}{dt}=\\
	& = \frac{d}{dt}\left(\frac{\partial \lagrangiano}{\partial \dot{\alpha}}\dot{\alpha}\right)\by{\ref{eq:energía-lagrangiana}}\frac{d\left(E+\lagrangiano\right)}{dt}=\frac{d E}{dt}+\frac{d\lagrangiano}{dt}=\frac{d E}{dt}
\end{align*}
Tenemos que la energía no varía con el tiempo.

    \section{Simetrías conservadas en física cuántica}\label{sec:simetrías-y-cantidades-conservadas-en-física-cuántica}

No podemos presentar con el mismo detalle que en el capítulo anterior, la relación entre simetrías y cantidades conservadas en mecánica cuántica, pues el marco matemático es completamente diferente, y tendríamos que desarrollar y explicar conceptos como los operadores, generadores de simetrías, campos, variedades diferenciales, grupos de Lie, etc. que no es el objetivo de este trabajo.

En mecánica cuántica, debemos rehacernos las mismas preguntas y seguir el mismo razonamiento realizado en los capítulos anteriores pues el marco matemático cambia, donde antes habían funciones, ahora hay operadores, los vectores no están en un espacio euclídeo sino en un espacio de Hilbert, y los escalares no son reales ($\R$) sino complejos ($\C$).

Como se desarrolla en~\cite{QMS}, las consideraciones sobre el lagrangiano~\eqref{eq:lagrangiano-clásico}, ecuación de Euler-Lagrange~\eqref{eq:euler-lagrange}, Teorema de Noether~\eqref{thm:noether}, se respetan y podemos hacer uso de ellos.
Además, tenemos los mismos resultados de simetrías y cantidades conservadas:

\begin{itemize}
	\item El sistema físico es homogéneo en el espacio $\Leftrightarrow$ Se conserva el momento.
	\item El sistema físico es isotrópico en el espacio $\Leftrightarrow$ Se conserva el momento angular.
	\item El sistema físico es homogéneo en el tiempo $\Leftrightarrow$ Se conserva la energía.
\end{itemize}

Sin embargo, en mecánica cuántica, obtenemos otras simetrías (y por tanto cantidades conservadas) que no están en física clásica y que han permitido conocer con más detalle el comportamiento de las partículas elementales.

Como ya se adelantó en la introducción, las simetrías vienen caracterizas por las matemáticas de los grupos, y al igual que se clasificaron las simetrías, tenemos grupos continuos y grupos discretos.
Y ¿porqué es importante ahora hablar de grupos y tal vez no antes?, pues porqué en física clásica, al trabajar con números reales, visualizar una simetría es sencillo, sin embargo, ahora al considerar las ecuaciones de la física cuántica tenemos que usar resultados en el campo de los complejos, y nos resultará mas sencillo trabajar con un lenguaje abstracto pero poderoso, es decir, trabajaremos con grupos.

\subsection{Spin}\label{subsec:spin}
Las rotaciones en el espacio en física clásica tienen asociado el grupo de simetrías $SO(3)$, al tratar las rotaciones en el campo complejo, dichas rotaciones tienen asociado el grupo de simetrías $SU(2)$.

Sabemos que el grupo $SU(2)$ tiene dimensión $3$ y que una base para este grupo es:
\begin{align}
	\sigma_x & = \begin{pmatrix} 0 && 1 \\ 1 && 0 \end{pmatrix} &
	\sigma_y & = \begin{pmatrix} 0 && -i \\ i && 0 \end{pmatrix} &
	\sigma_z & = \begin{pmatrix} 1 && 0 \\ 0 && -1 \end{pmatrix}
\end{align}

Las tres matrices anteriores son conocidas como matrices de Pauli~\autocite[50]{IQC} que son usadas para definir los operadores del momento del spin magnético.

Volviendo a la conservación de cantidades, la simetría rotacional conserva el operador momento angular, pero ahora dicho operador, que llamaremos $\hat{J}$ es en realidad una suma del análogo clásico de momento angular espacial ahora como operador, que llamaremos $\hat{L}$, más un nuevo operador asociado al spin que llamaremos $\hat{S}$.

El teorema de Noether, nos dice que bajo esta simetría, la carga conservada es $\hat{J}=\hat{L}+\hat{S}$, y de aquí obtenemos dos conclusiones muy curiosas~\autocite[45]{QMS}.
\begin{itemize}
	\item Cuando la rotación se estudia sobre una partícula descrita por un campo de escalares, para que se cumpla el teorema de Noether el spin debe valer $0$. La única partícula conocida que viene descrita como un campo de escalares es el bosón de Higgs, que efectivamente tiene spin $0$.
	\item Cuando la rotación se estudia sobre una partícula descrita por un campo de vectores, para que se cumpla el teorema de Noether el spin debe valer $1$. Ejemplos de estos campos son los que modelizan el fotón y el mesón, que efectivamente tienen spin $1$.
\end{itemize}

\subsection{Simetría en electromagnetismo}\label{subsec:simetría-en-electromagnetismo}
Vamos a partir directamente de las ecuaciones de Maxwell, sin deducir este resultado~\autocite[38]{ISMPP}:
\begin{align}
	\label{eq:maxwell}
	\nabla\cdot E & = \rho & \nabla\cdot B & = 0 \\
	\nabla\times E & = -\frac{\partial B}{\partial t} & \nabla\times B & = J+\frac{\partial E}{\partial t}
\end{align}
Donde $B$ es el campo magnético, $E$ es el campo eléctrico, $J$ es la densidad de carga eléctrica y $\rho$ es la carga eléctrica.

El lagrangiano de una partícula en un campo electromagnético es
\begin{equation}
	\label{eq:lagrangiano-electromagnético}
	\lagrangiano(q,\dot{q},t)=\frac{1}{2}m\dot{q}^2-\rho\phi(q, t)+\rho\dot{q}\cdot A(q,t)
\end{equation}
Donde $\phi$ es el potencial eléctrico y $A$ es el potencial magnético.

¿Y cuales son las transformaciones que dejan invariante las ecuaciones del movimiento?. En este caso, una transformación del tipo $q\longrightarrow q+\frac{df}{dt}$ para cualquier función diferenciable.

Este tipo de transformación define el flujo de Noether y la carga conservada que es $\nabla J=0$, es decir que la carga eléctrica no cambia con el tiempo, y por tanto que la carga eléctrica es una constante del sistema.

Si cuantizamos las ecuaciones de Maxwell, calculando el Hamiltoniano y aplicando la ecuación de Schrödinger $i\hbar\frac{\partial \Psi(q,t)}{\partial t}=H\Psi$, obtenemos unas ecuaciones, que a diferencia del caso clásico, no tienen simetría global ante rotaciones, sino que se convierten en simetrías locales, lo que se define como simetría gauge~\cite{MAQFT}.

Exactamente, la evolución temporal de una función de onda, al aplicarle la simetría gauge, nos modifica la solución de la ecuación de Schrödinger en un cambio de fase, $e^{i\alpha}$, que al ser unitario no afecta al módulo de la función de onda y por tanto el estado cuántico no se modifica.

Como la simetría viene caracterizado por un cambio de fase, que sólo tiene un parámetro, el grupo de simetrías asociado es $U(1)$ y el valor invariante por este grupo es un escalar que ya sabemos que es \textbf{la carga eléctrica}. Para conseguir que la simetría local se convierta en simetría global, es necesario añadir un campo escalar que compense y de esta manera aparece un nuevo campo que tiene las propiedades del \textbf{fotón}.

Hemos visto, como la simetría da explicaciones a propiedades fundamentales de la física, pero es que además, ahora, gracias a la simetría empezamos a ver el origen de las partículas elementales.

\subsection{Carga, paridad y tiempo}\label{subsec:paridad-carga-y-tiempo}
Las simetrías discretas en el espacio-tiempo, tales como paridad (o reflexión especular, simbolizada por P) e inversión temporal (T, representa la inversión de todos los movimientos), así como la simetría interna de conjugación partícula-antipartícula (C), aparecen exactas en la naturaleza cuando las únicas interacciones en juego son las fuertes y electromagnéticas. Sin embargo, todas ellas sufren una ruptura ordenada para fenómenos en que intervienen las interacciones débiles, como por ejemplo la desintegración de algunas partículas y núcleos atómicos, y algunos procesos de fusión nuclear como los que ocurren en las estrellas, incluyendo nuestro Sol~\autocite{IFIC}.

La violación de P fue descubierta en primer lugar en las desintegraciones beta de núcleos atómicos, y la de C y de P (por separado) en la desintegración del pión, en los años 50 del pasado siglo. La violación de CP fue descubierta inesperadamente en 1964 en la desintegración de kaones neutros, y se volvió a medir en las factorías de mesones B en 2001, mientras la demostración de la violación de T, usando el entrelazamiento de mesones B neutros, tuvo que esperar a 2012. En el Modelo Estándar se describen estas rupturas como consecuencia de dos facetas muy diferentes de la física: en el caso de P y C juega un papel fundamental la quiralidad, pues los campos levógiros y dextrógiros se transforman de forma distinta. Para CP y T es de suma importancia el contenido de partículas de la teoría, que necesita al menos tres familias de quarks y leptones para justificar la violación de esas simetrías~\autocite{IFIC}.

Hasta el momento la única simetría discreta que permanece exacta es la combinación CPT. El Teorema CPT, demostrado durante los años 50 del siglo XX, establece que una teoría cuántica de campos que describa interacciones en un espacio-tiempo plano, es necesariamente invariante bajo la transformación de simetría CPT si verifica estas tres condiciones:
\begin{itemize}
	\item Ser invariante Lorentz.
	\item Incluir sólo interacciones locales.
	\item Evolución unitaria.
\end{itemize}
~\autocite{IFIC}

\subsection{Otras simetrías}\label{subsec:otras-simetrías}
La teoría gauge mas exitosa en la actualidad es el \define{modelo estándar}{Modelo estándar}, que podemos caracterizar a través de sus grupos de simetrías $U(1)\times SU(2)\times SU(3)$.

Como ya hemos visto, la simetría en $U(1)$ conlleva a la conservación de la carga eléctrica. Y la ruptura de la simetría local nos define la existencia del fotón, como partícula portadora de la interacción.

Análogamente ocurre con el resto de simetrías, en el caso de $SU(2)$ que aparecía en las simetría rotacionales del campo magnético y que lleva a la conservación del momento angular total, tiene además importancia en el estudio de la interacción débil y de las conservaciones de los sabores~\autocite[107]{ISMPP}.

Según el modelo estándar de física de partículas, se define \define{sabor}{Sabor} al atributo que distingue a cada uno de los seis quarks: \textbf{u} (up, arriba), \textbf{d} (down, abajo), \textbf{s} (strange, extraño), \textbf{c} (charm, encantado), \textbf{b} (bottom, fondo) y \textbf{t} (top, cima).

Por último la simetría de $SU(3)$ caracteriza las interacciones fuertes y su carga conservada es la carga de color. Se define el \define{color}{Color} como una propiedad cuántica que distingue entre partículas que interaccionan con la fuerza nuclear fuerte.



    \printbibliography

    \printindex
\end{document}
