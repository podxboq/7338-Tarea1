\chapter{Mecánica Lagrangiana}\label{ch:mecanica-lagrangiana}
Para poder entender el Teorema de Noether, piedra angular en el estudio de las simetrías, es preciso conocer primero un concepto propio de la física denominada ``Mecánica Lagrangiana''.

A partir de la formulación de las leyes del movimiento de Newton, los sistemas físicos se estudiaban mediante la posición de los elementos que los componen (coordenadas x, y, z; ángulos y módulos en el caso de coordenadas polares; etc) y sus interacciones.
Sin embargo, esto suponía que su análisis se volviera demasiado complejo en determinadas situaciones.

Así, en 1788, el matemático Joseph-Louis Lagrange, desarrolló una manera de describir la mecánica que suponía una simplificación del estudio en gran cantidad de escenarios.

\section{El lagrangiano del sistema}\label{sec:el-lagrangiano-del-sistema}
En la mecánica lagrangiana, para un sistema, se considera el conjunto de $N$ parámetros que define la posición de todos sus elementos, estos parámetros reciben el nombre de \define{coordenadas generalizadas}{Coordenadas generalizadas}~\cite{GTP} y se denota por $q_i\ \forall\ i=1\dots N$.
Llamamos \define{velocidades generalizadas}{Velocidades generalizadas} a la derivada respecto del tiempo de las coordenadas generalizadas, que denotaremos como $\dot{q_i}\ \forall\ i=1\dots N$.

Vamos a restringirnos al caso unidimensional para simplificar la notación, podemos consultar el caso general en~\cite{Goldstein}.

Se define el \define{lagrangiano}{Lagrangiano} $\lagrangiano$, como la aplicación escalar que se forma mediante la energía cinética ($T$) y la energía potencial ($V$) de tal manera que:
\begin{equation}
	\label{eq:lagrangiano_clasico}
	\lagrangiano(q,\dot{q},t)=T-V=\frac{1}{2}m\dot{q}^2-V(q,t)\right)
\end{equation}

Un resultado interesante, y que usaremos más adelante, es expresar como depende la energía en función del lagrangiano, para ello observemos que:
\begin{equation*}
	\frac{\partial\lagrangiano}{\partial\dot{q}}\by{\ref{eq:lagrangiano_clasico}}m\dot{q}\so \frac{\partial\lagrangiano}{\partial\dot{q}}\dot{q} =m\dot{q}^2 = 2T
\end{equation*}

Por lo tanto:
\begin{equation}
	\label{eq:energia-lagrangiana}
	E=T+V=T+V+T-T=2T-\lagrangiano=\frac{\partial\lagrangiano}{\partial\dot{q}}\dot{q}-\lagrangiano
\end{equation}
\section{Acción}\label{sec:accion}

Un concepto muy importante en mecánica, asociado al Lagrangiano $\lagrangiano$, es el de \define{integral de acción}{Integral de acción} o simplemente \define{acción}{Acción}, que es un valor numérico calculado a través de la integral del lagrangiano sobre el tiempo entre dos instantes.
Si $\maps{\alpha}{I=[t_0, t_1]}{\R^n}$ es una función que representa el movimiento de una partícula, la acción se define como
\begin{equation}
	\label{eq:accion}
	S_\lagrangiano(\alpha) = \int_{I}\lagrangiano(\alpha(t), \dot{\alpha}(t), t)\ dt
\end{equation}

\begin{proposition}
	En los términos de la definición~\eqref{eq:accion}, el funcional $S_\lagrangiano$ es diferenciable en $\alpha$ y su valor es
	\begin{equation}
		\label{eq:accion_diferencial}
		dS_\lagrangiano(\alpha)(h)=\int_{I}\left( \frac{\partial \lagrangiano}{\partial\alpha}h+\frac{\partial \lagrangiano}{\partial\dot{\alpha}}\dot{h}\right) dt
	\end{equation}
\end{proposition}
\begin{proof}
	Por la definición de diferencial~\cite{FV} tenemos que para cualquier función $h$
	\begin{equation*}
		S_\lagrangiano(\alpha + h) -  S_\lagrangiano(\alpha) = dS_\lagrangiano(\alpha)(h) +\mathcal{O}(h^2)
	\end{equation*}
	Por otra parte y por la definición de la acción
	\begin{equation*}
		\begin{split}
			S_\lagrangiano(\alpha + h) -  S_\lagrangiano(\alpha) = & \int_{I}\lagrangiano(\alpha(t)+h(t), \dot{\alpha}(t)+\dot{h}(t), t)\ dt - \\
			- & \int_{I}\lagrangiano(\alpha(t), \dot{\alpha}(t), t)\ dt =\\
			=& \int_{I}\left( \frac{\partial \lagrangiano}{\partial\alpha}h+\frac{\partial \lagrangiano}{\partial\dot{\alpha}}\dot{h}\right)\ dt+\mathcal{O}(h^2)
		\end{split}
	\end{equation*}

	Donde en la última igualdad hemos usado el desarrollo de Taylor en una variable.
	Por lo tanto, igualando ambos términos, tenemos que:
	\begin{equation*}
		dS_\lagrangiano(\alpha)(h) =\int_{I}\left( \frac{\partial \lagrangiano}{\partial\alpha}h+\frac{\partial \lagrangiano}{\partial\dot{\alpha}}\dot{h}\right) dt
	\end{equation*}
\end{proof}

\section{Ecuación de Euler-Lagrange}\label{sec:ecuacion-de-euler-lagrange}
En la mayoría de los casos, a la hora de estudiar un sistema físico, podemos centrarnos exclusivamente en las variaciones (modificaciones) que ocurren en él mismo. Mediante el ``\define{Cálculo Variacional}{Cálculo Variacional}'', que es un campo de las matemáticas que se ocupa de estudiar un sistema funcional a través de los máximos y mínimos de sus funciones, podemos obtener la información necesaria para comprender los cambios de un sistema determinado.

El problema del llamado ``Principio Variacional'' surge en la historia de las matemáticas a través del estudio de la ``\define{Curva Braquistócrona}{Curva Braquistócrona}'' (del griego bráchystos, brevísimo y chronos, tiempo), planteado inicialmente por Johann Bernoulli en 1696~\cite{AE}.

Si consideramos un sistema sometido a fuerzas gravitatorias y en el que hay dos puntos a distinta altura, se llama Curva Braquistócrona, a aquella que hace que el tiempo que tarda un cuerpo inicialmente parado que está situado en el punto más alto (al que llamaremos A), en llegar al punto más bajo (B), sea mínimo.

Intuitivamente podría parecer que la función que cumple esto sería la línea recta, pero demostraremos a continuación que esto no es así. Veamos un esquema del problema:

\begin{figure}[H]
	\centering
	\begin{tikzpicture}
		\draw (0,0) -- (xyz cs:x=6);
		\draw (0,0) -- (xyz cs:y=6);
		\node (A) at (1,5) {A};
		\node (B) at (5,1) {B};
		\draw (1.1,5) -- (4.9,1);
		\draw (1.1,5) .. controls (3,4) .. (4.9,1);
		\draw (1.1,5) .. controls (3,1) .. (4.9,1);
	\end{tikzpicture}
	\caption{Ejemplo de posibles caminos entre dos puntos.}\label{pic:braquistocrona}
\end{figure}

En 1750, Leonhard Euler y Joseph-Louis Lagrange, encontraron una solución para esto.

\begin{theorem}
	Para que una función real $\alpha$ alcance un valor extremo (máximo o mínimo) en la acción $S_\lagrangiano$ se tiene que cumplir la siguiente igualdad llamada \define{ecuación de Euler-Lagrange}{Ecuación de Euler-Lagrange}
	\begin{equation}
		\label{eq:euler-lagrange}
		\frac{\partial \lagrangiano}{\partial \alpha}=\frac{d}{dt}\left( \frac{\partial \lagrangiano}{\partial \dot{\alpha}} \right)
	\end{equation}
\end{theorem}
\begin{proof}
	Para que la acción $S_\lagrangiano$ alcance en $\alpha$ un extremo, es necesario que su diferencial se haga cero en $\alpha$, es decir, que $dS_\lagrangiano(\alpha)(h)=0$ para cualquier función $h$.
	\begin{equation*}
		\begin{split}
			0 = dS_\lagrangiano(\alpha)(h)~\by{\ref{eq:accion_diferencial}}&\int_{I}\left( \frac{\partial \lagrangiano}{\partial\alpha}h+\frac{\partial \lagrangiano}{\partial\dot{\alpha}}\dot{h}\right) dt =\\
			= & \int_{I}\left( \frac{\partial \lagrangiano}{\partial\alpha}h+\frac{d}{dt}\left( \frac{\partial \lagrangiano}{\partial \dot{\alpha}}h \right)-\frac{d}{dt}\left( \frac{\partial \lagrangiano}{\partial \dot{\alpha}} \right)h\right)\ dt = \\
			= & \int_{I}\left( \frac{\partial \lagrangiano}{\partial \alpha}h-\frac{d}{dt}\left( \frac{\partial \lagrangiano}{\partial \dot{\alpha}} \right)h\right)\ dt +\int_{I}\frac{d}{dt}\left( \frac{\partial \lagrangiano}{\partial \dot{\alpha}}h \right)\ dt = \\
			= & \int_I\left( \frac{\partial \lagrangiano}{\partial \alpha}-\frac{d}{dt}\left( \frac{\partial \lagrangiano}{\partial \dot{\alpha}} \right)\right)h\ dt + \left[ \frac{\partial \lagrangiano}{\partial \dot{\alpha}}h \right]_{t_0}^{t_1}
		\end{split}
	\end{equation*}
	En particular si nos restringimos a las funciones $h$ que cumplen $h(t_0)=0=h(t_1)$, el segundo sumando vale $0$ y el para el primer sumando tenemos que
	\begin{equation*}
		\int_{I}\left( \frac{\partial \lagrangiano}{\partial \alpha}-\frac{d}{dt}\left( \frac{\partial \lagrangiano}{\partial \dot{\alpha}} \right)\right)h\ dt = 0 \so \frac{\partial \lagrangiano}{\partial \alpha}-\frac{d}{dt}\left( \frac{\partial \lagrangiano}{\partial \dot{\alpha}} \right) = 0
	\end{equation*}
\end{proof}

A partir de ahora, vamos a definir la \define{trayectoria}{Trayectoria} de una partícula, como la curva que cumple con las ecuaciones de Euler-Lagrange.

