\chapter{Mecánica Lagrangiana}\label{ch:mecanica-lagrangiana}

\section{El lagrangiano del sistema}\label{sec:el-lagrangiano-del-sistema}

La mecánica lagrangiana es una reformulación de la mecánica Newtoniana, introducida por Joseph-Louis de Lagrange en 1788, donde para cada partícula del sistema se consideran los valores que definen su posición, dichos valores reciben el nombre de \define{coordenadas generalizadas}{Coordenadas generalizadas}~\autocite{GTP}.
Llamamos \define{velocidades generalizadas}{Velocidades generalizadas} a la derivada respecto del tiempo de las coordenadas generalizadas.

En un sistema de $N$ partículas, tendremos para cada partícula, $n$ coordenadas generalizadas, denotamos por $q_i,\ i=1\dots N$ al vector formado por las coordenadas generalizadas de cada partícula y por $q=(q_1,\dots,q_N)$ al vector de las coordenadas generalidas del sistema.
Observamos que la dimensión de $q$ es $nN$.

Se define el \define{lagrangiano}{Lagrangiano} $\lagrangiano$, como la aplicación escalar que se forma mediante la energía cinética ($T$) y la energía potencial ($V$) de tal manera que:
\begin{equation}
	\label{eq:lagrangiano_clasico}
	\lagrangiano(q,\dot{q},t)=T-V=\sum\limits_{i=1}^{N} \frac{1}{2}m\|\dot{q_i}\|^2-V(q_i,t)
\end{equation}

\section{integral de acción}\label{sec:integral-de-accion}

Un concepto muy importante en mecánica, asociado al Lagrangiano $\lagrangiano$, es el de \define{integral de acción}{Integral de acción} o simplemente \define{acción}{Acción}, que es un valor numérico calculado a través de la integral del lagrangiano sobre el tiempo entre dos instantes.
Si $\maps{\alpha}{I=[t_0, t_1]}{\R^3}$ es una función que representa el movimiento de una partícula, la acción se define como
\begin{equation}
	\label{eq:accion}
	S_\lagrangiano(\alpha) = \int_{I}\lagrangiano(\alpha(t), \dot{\alpha}(t), t)\ dt
\end{equation}

\begin{proposition}
	En los términos de la definición~\eqref{eq:accion}, el funcional $S_\lagrangiano$ es diferenciable en $\alpha$ y su valor es
	\begin{equation}
		\label{eq:accion_diferencial}
		dS_\lagrangiano(\alpha)(h)=\int_{I}\left( \frac{\partial \lagrangiano}{\partial\alpha}h+\frac{\partial \lagrangiano}{\partial\dot{\alpha}}\dot{h}\right) dt
	\end{equation}
\end{proposition}
\begin{proof}
	Por la definición de diferencial tenemos que para cualquier funcion $h$
	\begin{equation*}
		S_\lagrangiano(\alpha + h) -  S_\lagrangiano(\alpha) = dS_\lagrangiano(\alpha)(h) +\mathcal{O}(h^2)
	\end{equation*}
	Por otra parte y por la definición de la acción
	\begin{equation*}
		\begin{split}
			S_\lagrangiano(\alpha + h) -  S_\lagrangiano(\alpha) = & \int_{I}\lagrangiano(\alpha(t)+h(t), \dot{\alpha}(t)+\dot{h}(t), t)\ dt - \\
			- & \int_{I}\lagrangiano(\alpha(t), \dot{\alpha}(t), t)\ dt =\\
			=& \int_{I}\left( \frac{\partial \lagrangiano}{\partial\alpha}h+\frac{\partial \lagrangiano}{\partial\dot{\alpha}}\dot{h}\right)\ dt+\mathcal{O}(h^2)
		\end{split}
	\end{equation*}

	Donde en la última igualdad hemos usado el desarrollo de Taylor en una variable.
	Por lo tanto, por la unicidad de diferencial de funciones, tenemos que:

	\begin{equation*}
		dS_\lagrangiano(\alpha)(h) =\int_{I}\left( \frac{\partial \lagrangiano}{\partial\alpha}h+\frac{\partial \lagrangiano}{\partial\dot{\alpha}}\dot{h}\right) dt
	\end{equation*}

\end{proof}

\section{Ecuación de Euler-Lagrange}\label{sec:ecuacion-de-euler-lagrange}

\begin{theorem}
	Una función real $\alpha$ alcanza un valor extremo (máximo o mínimo) en el funcional $S_L$ sí y sólo sí se da la siguiente igualdad llamada \define{ecuación de Euler-Lagrange}{Ecuación de Euler-Lagrange}
	\begin{equation}
		\label{eq:euler-lagrange}
		\frac{\partial \lagrangiano}{\partial \alpha}=\frac{d}{dt}\left( \frac{\partial \lagrangiano}{\partial \dot{\alpha}} \right)
	\end{equation}
\end{theorem}
\begin{proof}
	Una función alcanza un valor máximo o mínimo en el funcional $S_L$ cuando su diferencial se hace cero, es decir, que $dS_\lagrangiano(\alpha)(h)=0$ para cualquier función $h$.
	\begin{equation*}
		\begin{split}
			0 = dS_\lagrangiano(\alpha)(h)~\by{\ref{eq:accion_diferencial}}&\int_{I}\left( \frac{\partial \lagrangiano}{\partial\alpha}h+\frac{\partial \lagrangiano}{\partial\dot{\alpha}}\dot{h}\right) dt =\\
			= & \int_{I}\left( \frac{\partial \lagrangiano}{\partial\alpha}h+\frac{d}{dt}\left( \frac{\partial \lagrangiano}{\partial \dot{\alpha}}h \right)-\frac{d}{dt}\left( \frac{\partial \lagrangiano}{\partial \dot{\alpha}} \right)h\right)\ dt = \\
			= & \int_{I}\left( \frac{\partial \lagrangiano}{\partial \alpha}h-\frac{d}{dt}\left( \frac{\partial \lagrangiano}{\partial \dot{\alpha}} \right)h\right)\ dt +\int_{I}\frac{d}{dt}\left( \frac{\partial \lagrangiano}{\partial \dot{\alpha}}h \right)\ dt = \\
			= & \int_I\left( \frac{\partial \lagrangiano}{\partial \alpha}-\frac{d}{dt}\left( \frac{\partial \lagrangiano}{\partial \dot{\alpha}} \right)\right)h\ dt + \left[ \frac{\partial \lagrangiano}{\partial \dot{\alpha}}h \right]_{t_0}^{t_1}
		\end{split}
	\end{equation*}
	En particular si nos restringimos a las funciones $h$ que cumplen $h(t_0)=0=h(t_1)$, el segundo sumando vale $0$ y el para el primer sumando tenemos que
	\begin{equation*}
		\int_{I}\left( \frac{\partial \lagrangiano}{\partial \alpha}-\frac{d}{dt}\left( \frac{\partial \lagrangiano}{\partial \dot{\alpha}} \right)\right)h\ dt = 0 \so \frac{\partial \lagrangiano}{\partial \alpha}-\frac{d}{dt}\left( \frac{\partial \lagrangiano}{\partial \dot{\alpha}} \right) = 0
	\end{equation*}
\end{proof}

\section{La energía en función del lagrangiano}\label{sec:la-energia-en-funcion-del-lagrangiano}

Un desarrollo interesante, y que necesitaremos más adelante para entender la conservación de la energía, es expresar como depende la energía del sistema en función del lagrangiano, para ello, vamos a considera un sistema físico con una sóla partícula, pues la generación a $N$ partículas es directa, observemos que:
\begin{equation*}
	\frac{\partial\lagrangiano}{\partial\dot{q}} =m\dot{q}\so \frac{\partial\lagrangiano}{\partial\dot{q}}\dot{q} =m\dot{q}^2 = 2K
\end{equation*}

Por lo tanto:
\begin{equation}
	\label{eq:energia-lagrangiana}
	E=K+T=K+T+K-K=2K-\lagrangiano=\frac{\partial\lagrangiano}{\partial\dot{q}}\dot{q}-\lagrangiano
\end{equation}

