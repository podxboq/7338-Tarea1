\section{Principio de mínima acción}\label{sec:principio-de-minima-accion}

El principio de mínima acción surgió como una consecuencia de la mecánica clásica aplicada a la formulación lagrangiana.

Consideremos una masa en movimiento y su trayectoria.

Cuando interpretamos un sistema físico mediante las leyes de Newton, resulta directo el saber que trayectoria seguirá dicho objetos, ya que la misma, viene dada a través de las fuerzas a las que se ve sometido a lo largo del tiempo. Sin embargo, cuando abordamos le problema desde una perspectiva lagrangiana

El principio de mínima acción o principio de Hamilton, es un postulado básico de la física para describir la evolución a lo largo del tiempo del estado de movimiento de una partícula.

El \textbf{principio de mínima acción} o \textbf{principio de Hamilton}, dice que las partículas se mueven a traves de trayectorias que minimizan la acción del lagrangiano.
