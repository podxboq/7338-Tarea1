\chapter{Primer teorema de Noether}

\section{Flujo}

Para entender bien el primer teorema de Noether, y sobre todo, para poder entender la relación entre simetrías y conservación de cantidades, que se verán en los siguientes capítulos, es necesario presentar el concepto del flujo de Noether.

Tenemos un sistema físico, con un Lagrangiano $L$ y $\maps{\alpha,\beta}{I=[t_0, t_1]}{\R^3}$ dos funciones, tales que los lagrangianos para ambas funciones está relacionado por una tercera función real que cumpla $L(\beta, \dot{\beta}, t) = L(\alpha, \dot{\alpha}, t)+\dot{f}$.
Entonces las respectivas acciones están relacionadas por una constante, pues

\begin{equation}
	\label{eq:variacion-accion}
	\begin{split}
		S_L(\beta) & = \int_I L(\beta, \dot{\beta}, t) dt = \int_I L(\alpha, \dot{\alpha}, t)\ dt+\int_I \dot{f}dt=S_L(\alpha) +[f]_{t_0}^{t_1}
	\end{split}
\end{equation}

Y en este caso, la relación entre las acciones es

\begin{equation}
	\label{eq:variacion-accion-relacion}
	S_L(\beta) - S_L(\alpha) = f(t_1) - f(t_0)
\end{equation}

Como la diferencia entre las acciones es una constante, sus diferenciales son iguales, y por tanto si $\alpha$ es una trayectoria, también lo es $\beta$ y además tenemos que

\begin{equation}
	\label{eq:diferencial_trayectoria}
	\begin{split}
	[f]
		_{t_0}^{t_1} = & S_L(\beta) - S_L(\alpha) = S_L(\alpha + (\beta-\alpha))-S_L(\alpha)=dS_L(\alpha)(\beta-\alpha)=\\
		\by{\ref{eq:accion_diferencial}} & \int_{I}\left( \frac{\partial L}{\partial\alpha}(\beta-\alpha)+\frac{\partial L}{\partial\dot{\alpha}}(\dot{\beta}-\dot{\alpha})\right) dt = \\
		\by{\ref{eq:euler-lagrange}} & \int_{I}\left( \frac{d}{dt}\left(\frac{\partial L}{\partial\dot{\alpha}}\right)(\beta-\alpha)+\frac{\partial L}{\partial\dot{\alpha}}(\dot{\beta}-\dot{\alpha})\right) dt=\\
		= & \int_{I}\frac{d}{dt}\left(\frac{\partial L}{\partial\dot{\alpha}}(\beta-\alpha)\right) dt = \frac{\partial L}{\partial\dot{\alpha}}(\beta-\alpha) + \text{ cte }
	\end{split}
\end{equation}

Es decir, que para trayectorias cuyos lagrangianos se diferencia por la diferencial de una función, existe una constante $\ct{C}$ tal que
\begin{equation}
	\label{eq:flujo_noether_para_trayectorias}
	\frac{\partial L}{\partial\dot{\alpha}}(\beta-\alpha) = \ct{cte}
\end{equation}

\begin{definition}
	Dadas dos trayectorias $\maps{\alpha,\beta}{I}{\R^3}$, llamamos \define{flujo de Noether}{Flujo de Noether} de $\alpha$ a $\beta$ a
	\begin{equation}
		\label{eq:flujo_noether}
		\frac{\partial L}{\partial\dot{\alpha}}(\beta-\alpha)
	\end{equation}
\end{definition}

Llegado a este punto, ya estamos en condiciones de enunciar el primer teorema de Noether, cuya demostración no sino el desarrollo de este capítulo hasta llegar a la expresión~\ref{eq:flujo_noether_para_trayectorias}.

\begin{theorem}
	\label{thm:noether}
	Si en un sistema con lagrangiano $L$ dos trayectorias $\alpha$ y $\beta$ cumple que $L(\beta, \dot{\beta}, t) = L(\alpha, \dot{\alpha}, t)+\dot{f}$ entonces el flujo de Noether de $\alpha$ a $\beta$, permanece constante.
\end{theorem}